%% Based on a TeXnicCenter-Template by Tino Weinkauf.
%%%%%%%%%%%%%%%%%%%%%%%%%%%%%%%%%%%%%%%%%%%%%%%%%%%%%%%%%%%%%

%%%%%%%%%%%%%%%%%%%%%%%%%%%%%%%%%%%%%%%%%%%%%%%%%%%%%%%%%%%%%
%% HEADER
%%%%%%%%%%%%%%%%%%%%%%%%%%%%%%%%%%%%%%%%%%%%%%%%%%%%%%%%%%%%%
\documentclass[a4paper,twoside,10pt]{report}
% Alternative Options:
%	Paper Size: a4paper / a5paper / b5paper / letterpaper / legalpaper / executivepaper
% Duplex: oneside / twoside
% Base Font Size: 10pt / 11pt / 12pt


%% Language %%%%%%%%%%%%%%%%%%%%%%%%%%%%%%%%%%%%%%%%%%%%%%%%%
\usepackage[Danish]{babel} %francais, polish, spanish, ...
%\usepackage[T1]{fontenc}
\usepackage[ansinew]{inputenc}

%\usepackage{lmodern} %Type1-font for non-english texts and characters


%% Packages for Graphics & Figures %%%%%%%%%%%%%%%%%%%%%%%%%%
\usepackage{graphicx} %%For loading graphic files
%\usepackage{subfig} %%Subfigures inside a figure

%% Please note:
%% Images can be included using \includegraphics{Dateiname}
%% resp. using the dialog in the Insert menu.
%% 
%% The mode "LaTeX => PDF" allows the following formats:
%%   .jpg  .png  .pdf  .mps
%% 
%% The modes "LaTeX => DVI", "LaTeX => PS" und "LaTeX => PS => PDF"
%% allow the following formats:
%%   .eps  .ps  .bmp  .pict  .pntg


%% Math Packages %%%%%%%%%%%%%%%%%%%%%%%%%%%%%%%%%%%%%%%%%%%%
%\usepackage{amsmath}
%\usepackage{amsthm}
%\usepackage{amsfonts}


%% Line Spacing %%%%%%%%%%%%%%%%%%%%%%%%%%%%%%%%%%%%%%%%%%%%%
%\usepackage{setspace}
%\singlespacing        %% 1-spacing (default)
%\onehalfspacing       %% 1,5-spacing
%\doublespacing        %% 2-spacing


%% Other Packages %%%%%%%%%%%%%%%%%%%%%%%%%%%%%%%%%%%%%%%%%%%
\usepackage{a4wide} %%Smaller margins = more text per page.
\usepackage{fancyhdr} %%Fancy headings
\usepackage{hyperref} %make TOC linkable
\usepackage{fancyvrb} %centreret verbatim
\usepackage{paralist}		% for: \begin{compactitem}
%\usepackage{longtable} %%For tables, that exceed one page


%%%%%%%%%%%%%%%%%%%%%%%%%%%%%%%%%%%%%%%%%%%%%%%%%%%%%%%%%%%%%
%% Options / Modifications
%%%%%%%%%%%%%%%%%%%%%%%%%%%%%%%%%%%%%%%%%%%%%%%%%%%%%%%%%%%%%
%% Based on a TeXnicCenter-Template by Tino Weinkauf.
%%%%%%%%%%%%%%%%%%%%%%%%%%%%%%%%%%%%%%%%%%%%%%%%%%%%%%%%%%%%%

%%%%%%%%%%%%%%%%%%%%%%%%%%%%%%%%%%%%%%%%%%%%%%%%%%%%%%%%%%%%%
%% OPTIONS
%%%%%%%%%%%%%%%%%%%%%%%%%%%%%%%%%%%%%%%%%%%%%%%%%%%%%%%%%%%%%
%%
%% ATTENTION: You need a main file to use this one here.
%%            Use the command "\input{filename}" in your
%%            main file to include this file.
%%

%%%%%%%%%%%%%%%%%%%%%%%%%%%%%%%%%%%%%%%%%%%%%%%%%%%%%%%%%%%%%
%% OPTIONS FOR ITEMIZE
%%%%%%%%%%%%%%%%%%%%%%%%%%%%%%%%%%%%%%%%%%%%%%%%%%%%%%%%%%%%%
\renewcommand{\labelitemii}{$\circ$} % angiver typografien imtemize - niveau 2 til at være en tom cirkel

%%%%%%%%%%%%%%%%%%%%%%%%%%%%%%%%%%%%%%%%%%%%%%%%%%%%%%%%%%%%%
%% NEW COMMANDS FOR FORMATTING
%%%%%%%%%%%%%%%%%%%%%%%%%%%%%%%%%%%%%%%%%%%%%%%%%%%%%%%%%%%%%
\newcommand{\fil}[1]{#1} %formatering for filnavne
\newcommand{\prg}[1]{\textsc{#1}} %formatering for programnavne
\newcommand{\lang}[1]{#1} %formatering for programmeringssprog
\newcommand{\funk}[1]{\textbf{#1}} %formatering for funktionsnavne
\newcommand{\var}[1]{\textbf{\$#1}} %formatering for variabel/parameter navne
\newcommand{\lref}[1]{\texttt{#1}} %formatering for kodeeksempel linienumre
\newcommand{\xmlatt}[1]{\texttt{#1}} %formatering for xml element atributter
\newcommand{\xmlelm}[1]{\texttt{#1}} %formatering for xml element navne

%%%%%%%%%%%%%%%%%%%%%%%%%%%%%%%%%%%%%%%%%%%%%%%%%%%%%%%%%%%%%
%% OPTIONS FOR FANCYHEADER
%%%%%%%%%%%%%%%%%%%%%%%%%%%%%%%%%%%%%%%%%%%%%%%%%%%%%%%%%%%%%
\pagestyle{fancyplain} %til at lave overskrift på hver side
%\setlength{\parindent}{0pt} %indrykning ved ny sektion
\fancyhf{} % delete current header and footer
\fancyhead[OL]{\leftmark}
\fancyhead[OR]{\thepage}
\fancyhead[EL]{\thepage}
\fancyhead[ER]{\leftmark} %giver overskrift \rightmark giver subsection overskrift
\fancypagestyle{plain}{%
\fancyhead{} % get rid of headers on plain pages
\renewcommand{\headrulewidth}{0pt} % and the line
}



%%%%%%%%%%%%%%%%%%%%%%%%%%%%%%%%%%%%%%%%%%%%%%%%%%%%%%%%%%%%%
%% OPTIONS FOR HYPERREF
%%%%%%%%%%%%%%%%%%%%%%%%%%%%%%%%%%%%%%%%%%%%%%%%%%%%%%%%%%%%%
\hypersetup{
    colorlinks,
    citecolor=black,
    filecolor=black,
    linkcolor=black,
    urlcolor=black
}


%%Space between paragraphs: half the height of the small x
%\setlength{\parskip}{0.5ex}

%%Indent at the beginning of a paragraph: set to zero
%\setlength{\parindent}{0ex}

%%Spacing between lines: 1.5 times
%% ==> Consider using the package 'setspace' instead.
%\linespread{1.5}


 %You need a file 'options.tex' for this


%%%%%%%%%%%%%%%%%%%%%%%%%%%%%%%%%%%%%%%%%%%%%%%%%%%%%%%%%%%%%
%% DOCUMENT
%%%%%%%%%%%%%%%%%%%%%%%%%%%%%%%%%%%%%%%%%%%%%%%%%%%%%%%%%%%%%
\begin{document}

%% Title Page %%%%%%%%%%%%%%%%%%%%%%%%%%%%%%%%%%%%%%%%%%%%%%%
%% ==> Write your text here or include other files.

%% The simple version:
\title{Kompleks data og logik i databasen}
\author{Andreas Falk}
\date{28-01-2014} %%If commented, the current date is used.
\maketitle

%% The nice version:
%%% Based on a TeXnicCenter-Template by Tino Weinkauf.
%%%%%%%%%%%%%%%%%%%%%%%%%%%%%%%%%%%%%%%%%%%%%%%%%%%%%%%%%%%%%

%%%%%%%%%%%%%%%%%%%%%%%%%%%%%%%%%%%%%%%%%%%%%%%%%%%%%%%%%%%%%
%% Deckblatt
%%%%%%%%%%%%%%%%%%%%%%%%%%%%%%%%%%%%%%%%%%%%%%%%%%%%%%%%%%%%%
%%
%% ATTENTION: You need a main file to use this one here.
%%            Use the command "\input{filename}" in your
%%            main file to include this file.
%%

%\begin{titlepage}

\begin{center}

%\vspace*{1cm}
\Large
\textsc{XPath, XQuery og SQLXML}\\

\vspace{5cm}

%\LARGE
\textsc{Databaser for udviklere: kompleks data og logik i databasen\\[0.5\baselineskip]
af\\[0.5\baselineskip]
Andreas Falk\\
{\normalsize \textsc{}}}\\

\vspace{5cm}
\textsc{\28-01-2014}\\ %%Date - better you write it yourself.

\vspace{1cm}
\textsc{%Supervisors:\\
%Prof. V. Nachname\\
%Dr. V. Nachname
}\\

\vspace{1cm}
\textsc{AAU\\
%Faculty of ...\\
%Institute of ...
}\\

\end{center}

%\end{titlepage}
 %%You need a file 'titlepage.tex' for this.
%% ==> TeXnicCenter supplies a possible titlepage file
%% ==> with its templates (File | New from Template...).


%% Inhaltsverzeichnis %%%%%%%%%%%%%%%%%%%%%%%%%%%%%%%%%%%%%%%
\tableofcontents %Table of contents
\cleardoublepage %The first chapter should start on an odd page.
\pagestyle{fancy} %Now display headings: headings / fancy / ...


%% Chapters %%%%%%%%%%%%%%%%%%%%%%%%%%%%%%%%%%%%%%%%%%%%%%%%%
\chapter{Introduktion}
\label{chap:Introduktion}
\section{Form�l}
\label{sec:Formaal}
Denne rapport handler prim�rt om hvordan xml dokumenter kan valideres, uds�ttes for effektive foresp�rgelser og sameksistere med og i en SQL database. Alle emner bliver behandlet ved brug af et gennemg�ende eksempel, som l�bende uds�ttes for alle de forskellige teknologier som n�vnes herunder.
\begin{itemize}
	\item XML Skema
	\item DTD
	\item XPath
	\item XQuery
	\item SQL/XML
	\item Full text search
\end{itemize}

\subsection{Fremgangsm�de}
\label{subsec:Fremgangsmaade}
Da det er generelt er umuligt at behandle et XML dokument, hvis ikke strukturen er kendt, skal der laves en mindre analyse af dette. Det er hvad kapitel \ref{chap:analyse_xml_struktur} omhandler, hvor XQuery og XPath benyttes. I kapitel \ref{chap:XML_skema_for_resource} udnyttes det at strukturen for xml dokumentet nu er defineret, og der oprettes et XML skema for at dokumentere dette p� en standard m�de, og dermed ogs� muligg�re validering af dokumentet.  Kapitel \ref{chap:XML_i_SQL_DB} handler om hvordan XMl dokumenter kan indl�ses i en database, p� forskellige m�der, samt at data kan udtr�kkes fra en database, og leveres som XML direkte fra databasen, uden brug af ekstra applikationer.  *TODO* beskriv resten af kapitlerne

Men f�r der kan startes p� alt dette, kommer f�rst en beskrivelse af det xml dokument, som rapporten benytter som datakilde.

\subsection{XML dokumentet resource}
\label{subsec:XML_dokumentet_resource}
XML dokumentet er gemt i filen \fil{resource.xml}, som er en kopi af filen \fil{gemexport.xml} fra kursets hjemmeside. Filen er at finde i mappen Data i projektmappen. Omd�bningen skyldes at navnet gemexport ikke var sigende for indholdet. 

Dataene i XML dokumentet er en liste over uddannelsesresourcer, hvor der for hver resource er en m�ngde oplysninger. Der er ud over titel og beskrivelse ogs� oplysninger om udgiver, kategorier og hvorn�r resourcen er oprettet og meget andet.  

I resten af rapporten vil brugen af navnet resource b�de referere til XML dokumentet, som er gemt i filen resource.xml, og til den instans af BaseX databasen der oprettes ud fra filen resource.xml. I tilf�lde hvor betydningen ikke fremg�r af sammenh�ngen, vil betydningen blive pr�ciseret eksplicit.

%\begin{itemize}
%	\item Implementation af effektiv h�ndtering af xml-filen gemexport.xml.
%	\begin{itemize}
%		\item F�rste skridt er en analyse af strukturen, hvor resultatet er et xml skema, som kan validere vores xml fil. Dette g�res ved brug af BaseX,  hvor der oprettes en ny database p� grundlag af resource.xml.
%		\item Flere forskellige foresp�rgelser skrives i hhv. Xpath og Xquery,  som bl.a. viser kardinalitetten af  elementer i xml'en. Samt om elementer er oprettet, men er tomme elementer. 
%		\item Som udgangspunkt antages det at xml'en indeholder alle yderligheder der kan forekomme, s� f.eks. maks l�ngde for et tekst element kan udledes vha. Foresp�rgelser mod xml'en.
%		\item Der inds�ttes en reference til det nyoprettede skema, i den oprindelige xml, og skemaet valideres med xml plugin i Notepad++. Dette kan selvf�lgelig g�res med et hvilken som helst xml valideringsv�rkt�j.

%	\end{itemize}
%	\item Nu skal xml'en gemmes i en relationel database, og her kan der v�lges flere tilgangsm�der.
%	\begin{itemize}
%		\item At modellere xml strukturen vha. Den Entitets og relationelle model, og splitte(shredde) xml op og inds�tte den, s� vi p� den m�de kan tilg� data og manipulere med den som en hvilken som helst anden sql database.
%		\item vi kan v�lge at inds�tte hele vores xmlfil i en r�kke i en tabel, og dern�st prim�rt benytte Xpath foresp�rgelser til foresp�rgelser. Disse foresp�rgelser kan, og vil sikkert blive suppleret med brug af XmlSql funktioner for at strukturere resultatet af Xpath foresp�rgelserne yderligere.
%		\item Sidste mulighed er delvis shredding, hvor f.eks.  de enkelte resourcer splittes ud i enkelte r�kker, med enkelte vigtige attributer, id, evt titel, trulket ud i egne kolonner, for at simplificere filtrering p� disse attributter.
%	\end{itemize}
%	\item Indl�sning foreg�r i postgresql 9.3, som kun underst�tter Xpath 1.0, dvs. Ingen Xquery og enkelte funktioner fra nutidig Xpath kan ikke benyttes. Derudover mangler postgresql underst�ttelse af xmlsql funktionen xmltable,  som ellers er oplagt til mapping fra xml til relationelle tabeller. Hvorimod de implementerede publishing functions er implementeret ok.
%	\end{itemize}







 
 \chapter{Analyse af resource}
\label{chap:analyse_xml_struktur}

%%%%%%%%%%%%%%%%%%%%%%%%%%%%%%%%%%%%%%%%%%%%%%%%  Form�l %%%%%%%%%%%%%%%%%%%%%%%%%%%%%%%%%%%%%%%%%%%%%%%%%%%
\section{Form�l}
\label{sec:Formaal_xml_analyse}
Det er en stor opgave at gennemskue, og udlede strukturen af et xml dokument. Is�r for et komplekst dokument med mange elementer, og i tilf�ldet med dokumentet resource, som indeholder mange oplysninger for hele 1741 resourcer, kan strukturen ikke gennemskues ved en simpel genneml�sning af filen. 
Alternativet kan v�re at oprette en database i \prg{BaseX}, og ved at bruge en r�kke \lang{XQuery} forsp�rgelser kan der samles nok informationer om elementerne i XML'en, til at kunne beskrive strukturen. P� denne m�de kan \prg{BaseX} hj�lpe med at skabe overblik. Men det er stadig en manuel process, hvor der l�bende udvikles nye foresp�rgelser efterh�nden som man opbygger kendskab til XML dokumentet, og identificerer nye aspekter der skal unders�ges. 

Figur \ref{xquery:count_empty_image} viser hvordan det kan afg�res om der findes elementer \xmlelm{image} hvor alle dets underelementer er tomme. Hvis man �nsker at lave samme foresp�rgelse for elementet \xmlelm{itemDate}, s� kr�ver det at man skriver en ny foresp�rgelse, der minder utroligt meget om den der allerede er udviklet.
\begin{figure}[ht]
\centering
\begin{BVerbatim}
count(//resource/image[url ='' and 
                       caption ='' and 
                       alttext='' and 
                       sourceurl=''])
\end{BVerbatim}
\caption{Foresp�rgsel der er specielt udviklet til resource}
\label{xquery:count_empty_image}
\end{figure}

Selvom denne iterative fremgangsm�de vil virke, kan der istedet udvikles en mere generel funktion, som behandler dokumentet ud fra generelle betragtninger, og derfor kan bruges for alle XML dokumenter. S�dan en \lang{XQuery} funktion udvikles i dette projekt, s� de �nskede metadata for XML strukturen bliver returneret p� en strukturet vis, og det er udviklingen af den funktionalitet som dette kapitel hanlder om.

%%%%%%%%%%%%%%%%%%%%%%%%%%%%%%%%%%%%%%%%%%%%%%% Brug af BaseX %%%%%%%%%%%%%%%%%%%%%%%%%%%%%%%%%%%%%%%%%%%%%%%%%%%%%%%%%
\section{Brug af BaseX}

Til udvikling af foresp�rgelser benyttes XML databasen \prg{BaseX 7.7.2}, hvor der oprettes en ny database der tager udgangspunkt i filen \fil{resource.xml}. Der �ndres ikke p� indstillingerne for oprettelsen, hvilket betyder at der oprettes b�de Text og Attribut index, for at effektivisere foresp�rgelserne.

Koden til funktionerne der udvikles findes i filen \fil{Kode/XQuery/XQuery-analyse-funktion.xq}


%%%%%%%%%%%%%%%%%%%%%%%%%%%%%%%%%%%%%%%%%%%%%%% Design af analyse funktion %%%%%%%%%%%%%%%%%%%%%%%%%%%%%%%%%%%%%%%%%%%%%%%%
\section{Design af analyse funktioner}
\label{sec:DesignAfAnalyseFunktion}
Der er n�sten uendeligt mange aspekter at vurdere i forhold til struktur og datatyper, som kan v�re interresant at unders�ge. F�lgende er udvalgt som resultatet for analysefunktionen, da de i lang udstr�kning er tilstr�kkeligt til at kunne definere et XML skema.
\begin{compactitem}
	\item Element navn
	\item Underelementer.
	\item Kardinaliteten af et element, alts� hvor mange gange det kan optr�de i dokumentet.
	\item Hvis et element indeholder data, s� findes den maksimale l�ngde af datastrengen.
	\item Attributer.
\end{compactitem}
\vspace{1.0mm}
Figur \ref{code:struktur_metadata-xml} viser den XML struktur som de n�vnte oplysninger skal pr�senteres i. For hvert element i dokumentet der analyseres, oprettes der et element \xmlelm{element} i resultat XML'en. Informationer vedr�rende elementet angives som attributter, og informationer om elementets attributer som selvst�ndige \xmlelm{attribute} underelementer. Denne struktur for \xmlelm{element} gentages rekursivt for alle underelementerne i dokumentet der analyseres. 
\begin{figure}[ht]
\centering
\begin{BVerbatim}
<elements>
 <element name="xmlwithattributes" maxlength="" card="1:1">
  <attribute name="test" maxlength="1"/>
  <element name="node1" maxlength="4" card="1:1">
    <attribute name="part" maxlength="5"/>
  </element>
 </element>
</elements>

baseret p� dette:
<xmlwithattributes test="1">
  <node1 part="12345">test</node1>
</xmlwithattributes>
\end{BVerbatim}
\caption{Struktur for genereret metadata-xml, samt analyseret XML}
\label{code:struktur_metadata-xml}
\end{figure}

For \xmlatt{maxlength} g�lder det at den kun udfyldes for elementer med en text() node. Virkem�den for \xmlatt{maxlength} er ikke defineret for elementer med mixed-content, og da den p�g�ldende xml resource.xml ikke indeholder mixed-content, bliver dette tilf�lde ikke h�ndteret yderligere.
Attributten \xmlatt{card} udfyldes med notationene \verb|1:1, 0:1, 0:n(v), 1:n(v)| og d�kker de normale muligheder for kardinalitet. I parentesen angiver \verb|v|, den st�rste v�rdi der blev fundet for \verb|n|.

V�r opm�rksom p� at elementerne behandles i grupper baseret p� deres navn og den kontekst de optr�der i. Dvs. at kardinalitetten ikke unders�ges for et enkelt element \xmlelm{url}, men for alle elementer der har navnet 'url', i den p�g�ldende kontekst.
 
%%%%%%%%%%%%%%%%%%%%%%%%%%%%%%%%%%%%%%%%%%%%%%% Implementation af local:analyze %%%%%%%%%%%%%%%%%%%%%%%%%%%%%%%%%%%%%%%%%%%%%%%
\section{Implementation af analyse funktioner}
\label{Implementation_af_analyse_funktioner}

L�sningen struktureres ud i flere funktioner, svarende til en funktion pr. oplysning. Alle implementerede funktioner oprettes i det namespace der hedder local, og navngives derfor f.eks. \funk{local:analyze}, Men i resten af rapporten udelades namespace, og den funktion vil blive refereret til som \funk{analyze}.
\begin{compactitem}
	\item \funk{analyze} er start funktionen der kaldes for at f� analyseret en xml struktur.
	\item \funk{analyze2} kaldes af \funk{analyze}, og er en rekursiv funktion der opbygger resultatet. 
	\item \funk{analyze\_attributes} finder alle attributter for et element, samt den maksimale l�ngde af data fundet i attributten.
	\item \funk{analyze\_maxlength} finder den maksimale l�ngde af data fundet i elementet.
	\item \funk{analyze\_card} finder kardinaliteten af elementet
\end{compactitem}
\vspace{1.0mm}
I de f�lgende afsnit beskrives de enkelte funktioners implementation.

%%%%%%%%%%%%%%%%%%%%%%%%%%%%%%%%%%%%%%%%%%%%  subsection analyze  %%%%%%%%%%%%%%%%%%%%%%%%%%%%%%%%%%%%%%%%%%%%%%%%%%%%%%
\subsection{Funktion \funk{analyze}}
\label{funktion_analyze}
Funktionen \funk{analyze} begynder analysen ved input elementets b�rn, og returnerer derfor ingen oplysninger om selve input elementet. Det betyder at en analyse af et helt xml dokument, eller fragment, b�r startes med kaldet \verb|local:analyze(/)|, alts� med dokumentroden som parameter. Dette skyldes at f.eks. kardinalitets analysen kr�ver information om hvilken kontekst elementerne optr�der i.

Den eneste funktionalitet der er implementeret i denne funktion er at tilf�je det yderste \xmlelm{elements} element til resultatet, og initiering af de rekursive kald.

%%%%%%%%%%%%%%%%%%%%%%%%%%%%%%%%%%%%%%%%%%%  subsection analyze2  %%%%%%%%%%%%%%%%%%%%%%%%%%%%%%%%%%%%%%%%%%%%%%%
\subsection{Funktion \funk{analyze2}}
\label{funktion_analyze2}

Funktionen \funk{analyze2} er den rekursive funktion der reelt behandler xml strukturen. Implementationen er indsat i figur \ref{code:func_analyze2}, og beskrives i det f�lgende. 
\begin{figure}[ht]
\centering
\begin{BVerbatim}
01: declare function local:analyze2($elements as item()*)
02: as element()*
03: { 
04:    let $names :=  distinct-values($elements/*/name())
05:    for $el in $names 
06:    let $maxlength := local:analyze_maxlength($elements/*[name()=$el])
07:    let $card := local:analyze_card($elements,$el)
08:    return 
09:    <element name='{$el}' maxlength='{$maxlength}' card='{$card}'>
10:        {local:analyze_attributes($elements/*[name()=$el])}
11:        {local:analyze2($elements/*[name()=$el])}
12:    </element>
13: };
\end{BVerbatim}
\caption{Den rekursive funktion analyze2}
\label{code:func_analyze2}
\end{figure}

Alle tags med samme navn i den nuv�rende kontekst behandles og analyseres samlet, og derfor grupperes disse ved at opbygge en liste af unikke elementnavne i linie \lref{04}. For hvert af navnene gemmes resultatet af \funk{analyze\_maxlength} og \funk{analyze\_card} i to variable. Brugen af variable sker udelukkende for at �ge l�sbarheden af opbygningen af \xmlelm{element} i linie \lref{09}. Attributter behandles i linie \lref{10}, hvor der opbygges et \xmlelm{attribut} element for hver atribut fundet. Tilsidst foretages det rekursive kald til \funk{analyze2}, som s�rger for at opbygge \xmlelm{element} for det n�ste niveau.

Et eksempel kan m�ske bedst illustrere, hvad det er der leveres til funktionerne som input, for det er vigtigt at have  styr p� den del:

Hvis \funk{analyze2} kaldes med \xmlelm{root} elementet, kommer variablen \$names til at indeholde en liste med et element, nemlig strengen 'resource', da det er det eneste element der er under \xmlelm{root}. Ved kald til \funk{analyze\_maxlength}, kan man i linie \lref{06} se at input er givet ved en foresp�rgelse, som finder alle de underelementer til \xmlelm{root}, som hedder \$el, i vores tilf�lde er det strengen 'resource'. S� der er kun et kald til \funk{analyze\_maxlength} i dette eksmpel, til geng�ld er inputtet, m�ngden af alle \xmlelm{resource} elementerne. 

Funktionen skal jo netop indsamle generelle oplysninger p� tv�rs af alle 'ens' elementer, og dette kan kun ske, hvis man betragter m�ngder af elementer. 

De andre funktionskald, foreg�r p� samme m�de, p�n�r for \funk{analyze\_card}, der har behov for et andet input. Det beskrives n�rmere i afsnit \ref{subsec:analyze_card}. 

%%%%%%%%%%%%%%%%%%%%%%%%%%%%%%%%%%%%%%%%%%%  subsection analyze_attributes  %%%%%%%%%%%%%%%%%%%%%%%%%%%%%%%%%%%%%%%%%%%%%%%
\subsection{Funktion \funk{analyze\_attributes}}
\label{subsec:analyze_attributes}

Koden kan ses i figur \ref{code:func_analyze_attributes}. Ud fra m�ngden af elementer som bliver givet som input bliver der i linie \lref{04} fundet de unikke navne for hver attribut, og i line \lref{06} bliver l�ngden af hver atribut fundet, og tilsidst bliver den maksimale v�rdi fundet sat ind som en del af resultatet. 
\begin{figure}[ht]
\centering
\begin{BVerbatim}
01: declare function local:analyze_attributes($elements as item()*)
02: as element()*
03: { 
04:   for $att in distinct-values($elements/@*/name())
05:     return <attribute name='{$att}' maxlength=
06:       '{max($elements/@*[name() = $att]/string-length())}'/>
07: };
\end{BVerbatim}
\caption{Funktionen analyze\_attributes}
\label{code:func_analyze_attributes}
\end{figure}


%%%%%%%%%%%%%%%%%%%%%%%%%%%%%%%%%%%%%%%%%%%  subsection analyze_maxlength  %%%%%%%%%%%%%%%%%%%%%%%%%%%%%%%%%%%%%%%%%%%%%%%
\subsection{Funktion \funk{analyze\_maxlength}}
\label{subsec:analyze_maxlength}

Funktionen finder den maksimale l�ngde, p� n�jagtig samme m�de som funktionen \funk{analyze\_attributes}, dog returneres der kun en v�rdi hvis der findes text i elementet.
%\begin{figure}[ht]
%\centering
%\begin{BVerbatim}
%01: declare function local:analyze_maxlength($elements as item()*)
%02: as xs:string
%03: { 
%04:   if($elements/text()) 
%05:     then string(max($elements/text()/string-length())) 
%06:     else ''
%07: };
%\end{BVerbatim}
%\caption{Funktionen analyze\_maxlength}
%\label{code:func_analyze_maxlength}
%\end{figure}

%%%%%%%%%%%%%%%%%%%%%%%%%%%%%%%%%%%%%%%%%%%  subsection analyze_card %%%%%%%%%%%%%%%%%%%%%%%%%%%%%%%%%%%%%%%%%%%%%%%
\subsection{Funktion \funk{analyze\_card}}
\label{subsec:analyze_card}

Som n�vnt har denne funktion brug for mere input, da konteksten er vigtig for at kunne bestemme kardinaliteten. Hvis funktionen kaldes med en m�ngde \xmlelm{resource} elementer, kan kardinaliteten for \xmlelm{resource} ikke bestemmes, da det ikke kan afg�res hvilke elementer der har samme for�lder. 

Hvis vi kigger p� eksemplet fra afsnit \ref{funktion_analyze2}, betyder det at funktionen kaldes med \xmlelm{root} i \$parent og 'resource' i child\_name. En vigtig pointe er at der kaldes med hele m�ngden af alle de mulige for�ldre, og derfor kan funktionen ogs� bestemme om et element er valgfrit eller obligatorisk.

\begin{figure}[ht]
\centering
\begin{BVerbatim}
01: declare function local:analyze_card($parents as item()*,$child_name as xs:string)
02: as xs:string
03: { 
04:   let $child_count := $parents/count(*[name()=$child_name])
05:   let $min := min($child_count)
06:   let $max := max($child_count)
07:   let $result := 
08:     if($min = 1 and $max = 1) then '1:1' else
09:     if($min = 0 and $max = 1) then '0:1' else
10:     if($min = 0 and $max > 1) then concat('0:n','(',$max,')') else 
11:     if($min > 0 and $max > 1) then concat('1:n','(',$max,')') else 'ERROR' 
12:   return $result
};
\end{BVerbatim}
\caption{Funktionen analyze\_card}
\label{code:func_analyze_card}
\end{figure}

Koden i linie \lref{04} i figur \ref{code:func_analyze_card} t�ller antallet af underelementer, for hvert enkelt for�lder element. Derfor indeholder variablen \$child\_count en m�ngde af resultater af count, et resultat for hver for�lder. Den mindste og st�rste v�rdi kan nu findes, og bruges til at ops�ttes resultatet p� den �nskede form.

%%%%%%%%%%%%%%%%%%%%%%%%%%%%%%%%%%%%%%%%%%%  section konklusion %%%%%%%%%%%%%%%%%%%%%%%%%%%%%%%%%%%%%%%%%%%%%%%
\section{Konklusion}
\label{sec:konklusion_xml_analyse}
Funktionen der finder attributter har vist sig at have en enkelt mangel, da det ikke bestemmes om en attribut er valgfri, og derfor kan resultatet ikke benyttes til at definere et XML skema udfra. Men da analysen har vist at resource ikke indeholder attributter implementeres denne udvidelse ikke, da den eksisterende funktionalitet er d�kkende for det nuv�rende behov.

S� med disse funktioner er det nu muligt, p� en simpel og let m�de, at skabe et overblik over strukturen for et vilk�rligt XML dokument. Resultatet for dokumentet resource kan ses i bilag \ref{app:Genereret analyse af resource.xml}. 

\chapter{XML skema for resource}
\label{chap:XML_skema_for_resource}
\section{Form�l}
\label{sec:formaal_skemaet}
Det prim�re form�l med et xml skema, er muligheden for at validere et xml dokument, og p� den m�de kontrollere at det aftalte format er overholdt. Ofte kan disse skemaer, som i sig selv er xml dokumenter, ogs� bruges til at automatisk generering af kode til behandling af alle de xml dokumenter som er valide i forhold til skemaet. Dette sker typisk gennem tredjeparts software. 
Et xml skema skal beskrive 2 vigtige egenskaber for at opfylde det beskrevne behov. 
\begin{description}
	\item[Struktur] som beskriver forholdet mellem for�ldre og b�rn elementer. Her kommer complexType og sequence med flere i spil
	\item[Datatype] som angiver format eller v�rdis�t for det data der er gemt i elementerne. Her benyttes simpleType og restriction med flere.
\end{description}
Med analyse funktionen fra kapitel \ref{chap:analyse_xml_struktur}, er det nemt at komme igang med at definere et skema, som kan validere strukturen i xml dokumentet. Hvor godt resultatet fra analyse funktionen end er, s� er fokus for den prim�rt elementerne og deres indbyrdes forhold til hinanden. S� med den information er grundstrukturen i skemaet klarlagt.
Men resultatet indeholder ikke nogen information om datatyperne, der er sv�rt at fastsl� automatisk, da det typisk foruds�tter en del dom�ne viden. Datatyperne vil derfor i h�j grad blive udledt udfra deres navne og implicitte betydning. Nogle af begr�nsningerne der benyttes i skemaet, er tilf�jet udelukkende for at demonstrere flere forskellige typer af begr�nsninger, og vil m�ske ikke v�re korrekte i XML dokumentets virkelige dom�ne.

\section{Skemaet \fil{resource.xsd}}
\label{sec:skemaet}

Det endelige skema kan ses i filen \fil{resource.xsd}, samt i bilag XXX *TODO* reference til bilag med hele skemaet *TODO*. Skemaet er opbygget, ved en manuel gennemgang af analyse resultatet i bilag *TODO reference til bilaget*. Hvert element er behandlet, og en type med en passende defintion er oprettet i skemaet. Der er brugt navngivne typer i stor udstr�kning, da det �ger l�sbarheden v�sentligt. 
Elementerne i skemaet beskrives her, grupperet udfra hvordan de er defineret. 
\vspace{1.5mm}
\begin{compactitem}
  \item 
F�lgende elementer i dokumentet er defineret som komplekse, med en sekvens af underelementer:
  \begin{compactitem}
    \item \xmlelm{root} anonym type
		\item \xmlelm{resource} resourceType
		\item \xmlelm{itemdate} itemdateType
		\item \xmlelm{identifier} identifierType
		\item \xmlelm{publisher} publisherType
		\item \xmlelm{image} imageType
		\item \xmlelm{interestingfact} interestingfactType
		\item \xmlelm{subjects} subjectsType
		\item \xmlelm{subject} subejctType
		\item \xmlelm{resourcekeywords} resourcekeywordsType
		\item \xmlelm{keywords} anonym type
  \end{compactitem}
	\vspace{1.0mm}
G�ldende for en sekvens er at alle elementerne skal optr�de i den angivede r�kkef�lge. Dette er i mods�tning til en anden indikator all, hvor om det g�lder at elementerne kan optr�de i en vilk�rlig r�kkef�lge. 
For hvert element i sekvensen kan der angives kardinalitet, ved brug af indikatorerne minOccurs og maxOccurs, og her er brugt de v�rdier som analyse funktionen returnerede.
	
	\vspace{1.0mm}
	\item Disse elementer er defineret som simple typer, med en begr�nsing angivet i forhold til den valgte basis type:
	\begin{compactitem}
	  \item \xmlelm{title} titleType oprettes med basistypen string, og minLength og maxLength angives til hhv. 5 og 110. 
		\item \xmlelm{url} urlType oprettes med basistypen string, og med et meget simpelt pattern der matcher formattet for en url.
		\item \xmlelm{sourceurl} refererer til den navngivne type urlType, der er forklaret ovenfor.
	  \item \xmlelm{primary} primaryType oprettes med basistypen string, og en enumeration der angiver de valide v�rdier for elementet.
	\end{compactitem}
	\vspace{1.0mm}	
  Disse typer illustrerer meget godt brugen af begr�nsninger, hvor det i forvejen udbyggede typehieraki nemt kan udvides med mere specialiserede typer.
	
	\vspace{1.0mm}
  \item Disse elementer defineres ved brug af standard typer:
	\begin{compactitem}
 		\item \xmlelm{ID} positivInteger
    \item \xmlelm{description} string
		\item \xmlelm{category} string
		\item \xmlelm{subcategory} string
		\item \xmlelm{term} string
  	\item \xmlelm{text} string
		\item \xmlelm{caption} string
		\item \xmlelm{alttext} string
		\item \xmlelm{name} string
		\item \xmlelm{agency} string
		\item \xmlelm{recordcreated} date
		\item \xmlelm{placedonline} date
  \end{compactitem}
\end{compactitem}
\vspace{1.5mm}

\section{Konklusion}
\label{sec:Konklusion_xmlskema}
Med et skema defineret, kan xml strukturen valideres. Det g�res ved at tilf�je en reference til skemaet i rod elementet. I dette projekt er \prg{Notepad++} samt pluginet 'xml tools' brugt til at foretage valideringen. 

\begin{figure}[ht]
\centering
\begin{BVerbatim} 
<root xmlns:xsi="http://www.w3.org/2001/XMLSchema-instance" 
      xsi:noNamespaceSchemaLocation="resource.xsd">
\end{BVerbatim}
\caption{Reference til skema tilf�jet i dokumentroden}
\label{code:reference_til_skema}
\end{figure}

\noindent Ikke overaskende er strukturen valid i forhold til det nye skema.


\chapter{XML i SQL database}
\label{chap:XML_i_SQL_DB}
%%%%%%%%%%%%%%%%%%%%%%%%%%%%%%%%%%%%%%%%%%% section form�l %%%%%%%%%%%%%%%%%%%%%%
\section{Form�l}
Det er muligt at gemme xml dokumenter i en SQL database, og til form�let er der lavet en ny SQL datatype der hedder xml. Ved at gemme XML i en kolonne af denne type, kan der laves foresp�rgelser via \lang{XPath} og \lang{XQuery}, i sammenspil med en normal SQL foresp�rgelse. Denne funktionalitet kaldes ogs� SQL/XML.

XML dokumentet resource skal nu gemmes i en relationel database, n�rmere bestemt \prg{PostgreSQL~9.3}, som pt. kun underst�tter \lang{XPath 1.0} foresp�rgelser, og hovedsageligt har prioriteret udviklingen af publiseringsfunktionerne over funktioner som \funk{xmltable} som ellers er en del af SQL standarden. Dette f� betydning hvis man v�lger at ville shredde et XML dokument op, og inds�tte det i en relationel model.

F�r et XML dokument kan gemmes i en sql database, b�r man overveje hvilken strategi der skal f�lges. 
\begin{itemize}
	\item XML dokumentet gemmes i en enkelt celle i en tabel, uden der foretages nogen som helst bearbejdning, og SQL databasen benyttes udelukkende som et opbevaringssted.
	\item XML dokumentet splittes(shreddes), og inds�ttes i en datamodel der er modelleret til at passe med den oprindelige XML struktur. Man kan foretage enten en fuldst�ndig eller delvis opsplitning. 	
\end{itemize}
 I tilf�ldet med resource, kunne en delvis opsplitning v�re at hvert \xmlelm{resource} element inds�ttes i en r�kke best�ende  af v�rdien fra \xmlelm{ID} trukket ud, og lagt i en kolonne, samt en kolonne af typen xml, der indeholder xml fragmenten for den enkelte resource.
En fuld opsplitning vil betyder at der ikke gemmes XML, da al data er remodeleret.

Der er fordele og ulemper ved begge valg, og hvilken der passer til en specifik opgave afh�nger i stor grad af hvilke operationer der skal foretages p� data. Hvis der skal laves mange komplekse foresp�rgelser og mange opdateringer, s� er den relationele model en god solid l�sning. 

Men hvis der mere er tale om opbevaring af nogle dokumenter, hvor foresp�rgelserne henter det hele, eller dele af dokumentet, og det skal bruges som XML, s� er ingen eller delvis shredding en l�sning. 

Et af de bedste argumenter for ikke at shredde et dokument, er hvis det er fra en ekstern kilde, og/eller strukturen ofte bliver �ndret. Her vil en �ndring i de ber�rte XPath/XQuery foresp�rgelser v�re nemmere at foretage, end at skulle modellere den relationelle model om. 

I de n�ste afsnit vil der blive udviklet funktioner til at kunne gemme XML dokumentet resource i en relationel database. Det sker ved hj�lp af forskellige \lang{XPath} foresp�rgelser. Derefter vil SQL/XML pupliceringsfunktioner blive brugt til at opbygge et XML dokument med samme struktur, som dokumentet resource. Fordelen ved at implementere en genopbygning af den oprindelige struktur, er at der allerede findes et XML skema, som resultatet, kan valideres op imod.

%%%%%%%%%%%%%%%%%%%%%%%%%%%%%%%%%%%%%%%%%%% section design af resource i DB %%%%%%%%%%%%%%%%%%%%%%
\section{Design af XML i relationel database}
\label{sec:Design_af_XML_resource_rel_db}
I forbindelse med denne rapport v�lges den totale opsplitning af XML dokumentet til den relationelle datamodel, af den simple grund at det giver st�rst mulighed for at pr�ve kr�fter med SQL/XML og \lang{XPath}.

Ud fra XML skemaet for resource, kan man bruge kardinaliteten for elementerne som et udgangspunkt for hvilke entiteter der er behov for. For repeterende elementer bliver n�dt til at modeleres som egen entiteter. I figur \ref{fig:ER_diagram} kan den model som er fremkommet ses. Selve modeleringen vil ikke blive gennemg�et, da den er af sekund�r betydning i denne rapport.
\begin{figure}[ht]
  \centering
   \framebox{\includegraphics[width=0.75\textwidth]{pic/gemexport_ER.pdf}}
   \caption{ER diagram for resource database}
   \label{fig:ER_diagram}
\end{figure}

%%%%%%%%%%%%%%%%%%%%%%%  subsection shredding funktioner  %%%%%%
\subsection{Funktioner til opsplitning af XML}
\label{subsec:design_shredding}
Grundideen er at behandle en resource ad gangen. Dvs. der skal implementeres en funktion der kan udtr�kke de data der skal gemmes i tabellen resource, og dern�st foretage inds�ttelsen.

For hvert af de underelementer til resource der er modeleret som sin egen entitet, skal der implementeres yderlige funktioner, der kan behandle disse xml fragmenter. 

Hver af disse funktioner, skal kontrollere at der rent faktisk finde data i XML strukturen, der kan retf�rdig�re at en r�kke oprettes. Der accepteres ikke r�kker fyldt med kun blanke felter.

Der skal derfor implementeres f�lgende funktioner til opsplitning:
\begin{compactitem}
	\item shredResources. Hovedfunktion der l�ser XML, og finder resource elementerne.
	\item shredResource. Opsplitter et resource element.
	\item shredImage. Opsplitter et image element.
	\item shredInterestingFact. Opsplitter et Interestingfact element.
	\item shredResourceKeywords. Opsplitter et Resourcekeywords element.
	\item shredSubject. Opsplitter et subject element.
\end{compactitem}
\vspace{1.0mm}

\subsection{Funktioner til generering af XML}
\label{subsec:design_generate_xml}
Der skal udvikles en funktion der kan opbygge en XML struktur der indeholder de samme elementer som det oprindelige XML dokument. Det er et krav at det nye dokument kan valideres udfra det XML skema der blev defineret i kapitel \ref{chap:XML_skema_for_resource}. For at opbygge dokumentet er det n�dvendigt at tage h�jde for at alle de tomme elementer i resource, der ikke bliver oprettet i tabellerne. 
F�lgende funktioner skal implementeres for at kunne opbygge dokumentet.
\begin{compactitem}
	\item reconstructResources. 
\end{compactitem}
\vspace{1.0mm}

%%%%%%%%%%%%%%%%%%%%%%%%%%%%%%%%%%%%%%%%%%% section implementation design af resource i DB %%%%%%%%%%%%%%%%%%%%%%
\section{Implementation XML i relationel database}
\label{sec:impl_af_XML_resource_rel_db}
For at kunne begynde indl�sning af xml i en database, skal tabellerne oprettes, og der skal v�re adgang til XML dokumentet.

Derfor er der oprettet f�lgende scritps til at oprette tabeller, og indl�se XML dokumentet i databasen:
\begin{compactitem}
	\item \fil{create\_tables.sql} opretter alle n�dvendige tabeller.
	\item \fil{insert xml into resource\_xmltable.sql} foretager hardcoded insert af xml dokumentet i tabel resource\_xml. Ved at indl�se XML'en p� denne m�de, undg�s problemer med rettigheder til at l�se fra filsystemet.
\end{compactitem}
\vspace{1.0mm}

\noindent I det efterf�lgende vil de implementerede funktioner til opsplitning af XML og generering af XML blive gennemg�et, og koden er at finde i disse scripts:
\begin{compactitem}
	\item \fil{extract\_resources\_from\_xml.sql} definerer alle shredding funktioner, og eksekverer dem.
	\item \fil{reconstruct\_resource.sql} definerer XML genereringsfunktionen, og eksekverer den.
\end{compactitem}
\vspace{1.0mm}

\noindent Alle scripts findes i mappen Kode/SQL

%%%%%%%%%%%%%%%%%%%%%%%  subsection funktioene shredResources  %%%%%%
\subsection{Funktionen shredResources}
\label{subsec:sql_shredResources}
Funktionen skal kaldes med en parameter af typen XML, n�rmere bestemt det dokument der er blevet indl�st i tabellen resource\_xml. Funktionen starter med at slette al indhold i tabellerne, som der skal l�ses data ind i. Efter indl�sning udskrives der lidt statistik for oprettelsen. 

Den centrale del af funktionen er vist i figur \ref{code:shredResources}, hvor linie \verb|05-10| er  en \lang{Xpath} foresp�rgelse, som finder alle de enkelte \xmlelm{resource} elementer. Et loop genneml�ber disse og funktionen shredResource kaldes for hver resource.
\begin{figure}[ht]
\centering
\begin{BVerbatim}
01: create function shredResources(root xml) returns void
02: ...
03: resources xml[];
04: ...
05: select 
06: 	xpath(
07: 	'/root/resource'
08: 	,root
09: 	) as data
10: into resources;
11: 
12: idx = 1;
13: loop
14:   --v�lg det aktuelle <resource>...</resource> element
15:   resourceXml = resources[idx];
16:   perform shredResource(resourceXml);
17:   idx = idx + 1;
18: exit when idx > array_length(resources,1);
19: end loop;
20: ...
\end{BVerbatim}
\caption{Centrale dele af funktionen shredResources}
\label{code:shredResources}
\end{figure}


%%%%%%%%%%%%%%%%%%%%%%%  subsection funktioen shredResource  %%%%%%%%%%
\subsection{Funktionen shredResource}
\label{subsec:sql_shredResource}
Figur \ref{code:shredResource} viser udvalgte dele af koden for shredResource. Behandlingen er opdelt i to loops, da der skal v�re oprettet en r�kke i resource, f�r der kan oprettes r�kker i de andre tabeller, da der er oprettet RI constraints, og figuren viser kode vedr�rende det f�rste loop.
I linie \verb|06| udv�lges alle underelementer, og dette array genneml�bes  efterf�lgende i et loop, hvor XPath bruges til at finde navnet p� elementet (linie \verb|11|). En case statement bruges til at behandle de forskellige elementnavne, hvor kun to af de mulige vises i figur \ref{code:shredResource}. Funktionen xmlAsText, som er en hj�lpefunktion skrevet til at lave typen xml om til typen text, kaldes for hvert felt. Linie \verb|15-16| viser hvordan XPath bruges til at f� fat i elementet \xmlelm{ID}, og gemmer indholdet i den tilh�rende variabel. I tilf�ldet med \xmlelm{publisher} kan det ses at der udtr�kkes 2 v�rdier fra hhv. \xmlelm{name } og \xmlelm{agency}. Det skyldes at publisher ikke er modeleret som sin egen entitet, og de to elementer fra dokumentet 'flyttes' til resource, s� det ekstra niveau kan undg�es. 

Det f�rste loop behandler kun de elementer, der skal inds�ttes p� tabellen resource, og n�r arrayet resourceChildren er l�bet igennem, kan r�kken inds�ttes. Dette er alm. SQL og er udeladt i figuren.

Nu kan alle de underelementer der er blevet modelleret som deres egne entiteter behandles, og dette foreg�r i endnu et loop, som opbygges p� samme m�de som loop 1, men hvor der, istedet for at udtr�kke v�rdier, kaldes funktioner der kan behandle de underliggende XML strukturer. 

\begin{figure}[ht]
\centering
\begin{BVerbatim}
01: create function shredResource(resourceXml xml) returns void 
02: ...
03: resourceChildren xml[];
04: tagname xml[];
05: ...
06: select xpath('child::*',resourceXml) into resourceChildren;
07: 
08: --f�rste loop finder alle v�rdier der skal bruges i resource tabellen
09: idy = 1;
10: loop
11: select xpath('name()',resourceChildren[idy]) into tagname;
12: 
13: case tagname[1]::text
14: when 'ID' then
15:   select xmlasText(xpath('/ID',
16: 	                 resourceChildren[idy]))::integer into resourceId;
17: ...
18: when 'publisher' then
19:   select xmlasText(xpath('/publisher/name',
20: 	                 resourceChildren[idy])) into publisherName;
21:   select xmlasText(xpath('/publisher/agency',
22: 	                 resourceChildren[idy])) into publisherAgency;
23: else
24: -- do nothing. resten af elementerne behandles i loop 2
25: end case;
26: idy = idy +1;
27: exit when idy > array_length(resourceChildren,1);
28: end loop;
29: ...
\end{BVerbatim}
\caption{Centrale dele af funktionen shredResource}
\label{code:shredResource}
\end{figure}

%%%%%%%%%%%%%%%%%%%%%%%  subsection de resterende funktioner  %%%%%%%%%%
\subsection{De resterende funktioner til opsplitning}
\label{subsec:sql_resterende _funkt}
Funktionerne til behandling af resten af xml strukturen er opbygget p� n�jagtig samme m�de, og vil ikke blive gennemg�et i detaljer. Der er sm� forskelle, da bl.a. \xmlelm{subjects} har flere niveauer, s� der skal benyttes et ekstra loop. 

Det �nskes ikke at oprette r�kker p� nogle af disse tabeller, hvis ikke der er fundet data til dem, og derfor er der kodet et ekstra tjek for indhold inden underelementerne behandles. Et eksempel taget fra shredInterestingFact vises i figur \ref{code:shredInterestingFact}. S� hvis blad elementerne kun indeholder blank, s� oprettes der ikke en forekomst i den p�g�ldnende tabel.

Her en liste over disse funktioner:
\begin{compactitem}
\item shredImage
\item shredInterestingFact 
\item shredResourceKeywords
\item shredSubject
\end{compactitem}
\vspace{1.0mm}

\begin{figure}[ht]
\centering
\begin{BVerbatim}
select xpath('/interestingfact[not(url ="" and text ="")]
              /child::*',fact) into factChildren;
\end{BVerbatim}
\caption{Tjek for indhold i tags}
\label{code:shredInterestingFact}
\end{figure}

\clearpage
%%%%%%%%%%%%%%%%%%%%%%%  subsection reconstruc  %%%%%%%%%%
\subsection{Funktionen reconstructResources}
\label{subsec:sql_reconstructResources}
Denne funktion benytter sig af de publiserings funktioner der er tilr�dighed, bl.a.
\begin{compactitem}
\item xmlroot
\item xmlattributes
\item xmlelement
\item xmlforest
\item xmlagg
\end{compactitem}
\vspace{1.0mm}

I figur \ref{code:Eksempel_paa_kodeXML} er starten af koden indsat, og det fremg�r tydeligt at SQL/XML ikke er s�rligt overskueligt ved st�rre dokumenter. Som det ses i linie \verb|24| og \verb|27| refereres her til XML som er opbygget i ikke viste subselects. Dette er n�dvendigt, da disse er opbygget med xmlagg funktionen, som ikke kan nestes i en enkelt foresp�rgelse.

Selv om det ikke er vist her, skal det bem�rkes at der benyttes left join og coalesce funktionen for at h�ndtere de tilf�lde hvor der f.eks. ikke findes en r�kke i interestingFacts tabellen. 
\begin{figure}[ht]
\centering
\begin{BVerbatim}
01: create function reconstructResources() returns xml
02: ...
03: --xmlroot(
04: -- root
05: xmlelement(name root,
06: xmlattributes('http://www.w3.org/2001/XMLSchema-instance' as "xmlns:xsi", 
07:               'resource.xsd' as "xsi:noNamespaceSchemaLocation"),
08: xmlagg(
09: -- resource
10:    xmlelement(name resource,
11: 
12: -- ID, title,description
13:       xmlforest(resourceid as "ID", Title, description),
14: -- itemdate
15:       xmlelement(name itemdate,
16:          xmlforest(recordcreated,placedonline)
17:       ),
18: ...
19: -- interestingfact
20:       xmlelement(name interestingfact,
21:          xmlforest(facturl as url,facttext as text)
22:       ),
23: -- resourcekeywords
24:       keywordxml      
25:       ,
26: -- subjects
27:       subjectxml      
28:    )))
29: --, version '1,0')
30: ...
\end{BVerbatim}
\caption{Eksempel p� kode der opbygger XML}
\label{code:Eksempel_paa_kodeXML}
\end{figure}

%%%%%%%%%%%%%%%%%%%%%%%  subsection resultat for shredding  %%%%%%%%%%
\section{Resultat for shredding}
\label{sec:res_shred}
Funktionen shredResources udskriver en opsummering af resultatet efter udf�relsen, og det kan ses i figur \ref{code:resultatshredResource}
\begin{figure}[ht]
\centering
\begin{BVerbatim}
INFO:  Sletter alt fra tabellerne: resource, image, 
                  interestingfact, keyword, subject
INFO:  Indl�ser data fra xml
INFO:  Indsat 1741 records i tabel: resource
INFO:  Indsat 1592 records i tabel: image
INFO:  Indsat 629 records i tabel: interestingfact
INFO:  Indsat 2546 records i tabel: keyword
INFO:  Indsat 3628 records i tabel: subject
INFO:  Indsat i alt 10136 records

Total query runtime: 5662 ms.
1 row retrieved.
\end{BVerbatim}
\caption{Resultat af shredResources}
\label{code:resultatshredResource}
\end{figure}

Nu er xmldokumentet fuldst�ndig normaliseret og tilg�ngelig i en SQL database.




%%%%%%%%%%%%%%%%%%%%%%%  subsection de resterende funktioner  %%%%%%%%%%
\section{Resultat for xml generering}
\label{sec:res_generering}

Det genererede xml dokument er eksporteret til filen \fil{reconstructed.xml} som kan findes i mappen Data. Dokumentet er valideret med \prg{Notepad++}. 

P� samme m�de som hele dette dokument blev konstrueret, kan funktionerne benyttes til at omforme resultatet fra en hvilken som helst forsp�rgelse, til XML format, og kan selvf�lgelig ogs� benyttes hvis der var foretaget delvis shredding af XML dokumentet.


\chapter{Full Text Search}
\label{chap:fulltextserach}
\section{Form�l}
\label{sec:fulltextsearchformaal}
Det skal v�re muligt at lave fornuftig s�gning p� beskrivelser af resourcer.
Full text search er en teknik hvor det man s�ger i, og selve foresp�rgelsen bearbejdes inden et resultat kan pr�senteres. Denne bearbejdning er det der g�r at s�gningen i figur \ref{fts:simpel_eksempel} returnerer true, selvom ordet 'searches' ikke er i teksten der s�ges i.
En del af denne bearbejdning foreg�r ved at mappe en masse forskellige former af samme ord til et ord, f.eks. huse og hus, bliver betragtet som det samme. Der ud over er der en masse ord, som f.eks. og, da, n� som ikke giver mening at s�ge p� da de typisk optr�der i store m�ngder i teksterne. Disse stopord ignoreres.  

\begin{figure}[ht]
\centering
\begin{BVerbatim}
select 
  to_tsvector('english','This is a test of full text search') 
  @@ 
  to_tsquery('english', 'searches');
------------------------------------
	                     t
\end{BVerbatim}
\caption{Simpelt eksempel p� full text search}
\label{fts:simpel_eksempel}
\end{figure}

Der implementeres full text search p� feltet resource.description i den modellerede SQL udgave af resource dokumentet. Det g�res ved at implementere en funktion der tager et ord som input, og s� returnere udvalgte kolonner fra de resourcer, hvor ordet er fundet. 


\section{Implementation af full text search}
\label{sec:impl_af_FTS}

Ved udf�relsen af en s�gning, skal begge funktioner to\_tsvector og to\_tsquery udf�res, og dette kan, med fordel, optimeres i tilf�lde hvor der skal s�ges ofte, eller i meget store dokumenter.

For at g�re det muligt at indeksere teksten, kan et indeks oprettes, hvor det er funktionen to\_tsvector der indekseres. En anden fremgangsm�de er at tilf�je en kolonne til tabellen, som kan indeholde to\_tsvector resultatet af det eller de felter der skal kunne s�ges i. I projektet er den sidste fremgangsm�de valgt. Der er fordele og ulemper ved begge fremgangsm�der, og ved den der er valgt her, er der ekstra pladsforbrug til at gemme ts\_vectoren og at den skal holdes opdateres hver gang description feltet opdateres. Og det er netop p� grund af behovet for denne synkronisering at metoden er valgt, for det er en oplagt mulighed for at implementere en trigger. Og selvom Postgresql har en funktion indbygget 'tsvector\_update\_trigger' som man kan angive som trigger, implementeres funktionen fra grunden.

S� f�lgende opgaver skal l�ses:

\begin{compactitem}
\item En triggerfunktion skal kodes. Den skal opdatere v�rdien af ts\_vectoren, s� den svarer til description feltet.
\item Tabellen resource skal udvides med den nye kolonne.
\item En trigger skal oprettes, som anvender den ny udviklede triggerfunktion.
\end{compactitem}
\vspace{1.0mm}
En simpel funktion implementeres, hvor funktionen to\_tsvector benyttes p� description, og tildeles til feltet FTS\_index\_col.
\begin{figure}[ht]
\centering
\begin{BVerbatim}
drop function if exists resource_trigger();
create function resource_trigger() returns trigger
as $$
declare
begin
  new.FTS_index_col := 
  to_tsvector('english', new.description);
return new;
end
$$
language plpgsql;
\end{BVerbatim}
\caption{Triggerfunktion}
\label{fts:trigger_func}
\end{figure}

Triggerfunktionen skal nu kobles til en trigger, og det g�res med f�lgende stykke kode. 
\begin{figure}[ht]
\centering
\begin{BVerbatim}
create trigger resource_FTS_index_col 
  before update of description,title or insert 
  on resource 
  for each row
  execute procedure resource\_trigger();
\end{BVerbatim}
\caption{Oprettelse af trigger}
\label{fts:trigger_create}
\end{figure}

Med alt dette defineret, plus et indeks oprettet p� den nye kolonne, kan der foresp�rges p� indholdet i description feltet. Funktionen searchDescription returnerer id, title og selve beskrivelsen for alle de r�kker der matcher det angivede ord.
\begin{figure}[ht]
\centering
\begin{BVerbatim}
drop function if exists searchDescription(text);
create function searchDescription(word text) returns 
  TABLE(resourceid int,title text, description text)
as $$
declare
begin
return query 
select r.resourceid,r.title,r.description
FROM resource as r
WHERE fts_index_col @@ to_tsquery('english', word);
end
$$
language plpgsql;
\end{BVerbatim}
\caption{Funktionen searchDescription}
\label{fts:searchDescription}
\end{figure}












%\chapter{The End}
%%%%%%%%%%%%%%%%%%%%%%%%%%%%%%%%%%%%%%%%%%%%%%%%%%%%%%%%%%%%%
%% BIBLIOGRAPHY AND OTHER LISTS
%%%%%%%%%%%%%%%%%%%%%%%%%%%%%%%%%%%%%%%%%%%%%%%%%%%%%%%%%%%%%
%% A small distance to the other stuff in the table of contents (toc)
%\addtocontents{toc}{\protect\vspace*{\baselineskip}}

%% The Bibliography
%% ==> You need a file 'literature.bib' for this.
%% ==> You need to run BibTeX for this (Project | Properties... | Uses BibTeX)
%\addcontentsline{toc}{chapter}{Bibliography} %'Bibliography' into toc
%\nocite{*} %Even non-cited BibTeX-Entries will be shown.
%\bibliographystyle{alpha} %Style of Bibliography: plain / apalike / amsalpha / ...
%\bibliography{literature} %You need a file 'literature.bib' for this.

%% The List of Figures
%\clearpage
%\addcontentsline{toc}{chapter}{List of Figures}
%\listoffigures

%% The List of Tables
%\clearpage
%\addcontentsline{toc}{chapter}{List of Tables}
%\listoftables


%%%%%%%%%%%%%%%%%%%%%%%%%%%%%%%%%%%%%%%%%%%%%%%%%%%%%%%%%%%%%
%% APPENDICES
%%%%%%%%%%%%%%%%%%%%%%%%%%%%%%%%%%%%%%%%%%%%%%%%%%%%%%%%%%%%%
\clearpage \thispagestyle{empty}
\begin{appendix}
\chapter{Genereret analyse af resource.xml}
\label{app:Genereret analyse af resource.xml}
\begin{BVerbatim}
<elements>
 <element name="root" maxlength="" card="1:1">
  <element name="resource" maxlength="" card="1:n(1741)">
   <element name="ID" maxlength="4" card="1:1"/>
   <element name="title" maxlength="107" card="1:1"/>
   <element name="description" maxlength="500" card="1:1"/>
   <element name="itemdate" maxlength="" card="1:1">
    <element name="recordcreated" maxlength="10" card="1:1"/>
    <element name="placedonline" maxlength="10" card="1:1"/>
   </element>
   <element name="identifier" maxlength="" card="1:1">
    <element name="url" maxlength="195" card="1:1"/>
   </element>
   <element name="publisher" maxlength="" card="1:1">
    <element name="name" maxlength="68" card="1:1"/>
    <element name="agency" maxlength="58" card="1:1"/>
   </element>
   <element name="image" maxlength="" card="1:1">
    <element name="url" maxlength="71" card="1:1"/>
    <element name="caption" maxlength="95" card="1:1"/>
    <element name="alttext" maxlength="200" card="1:1"/>
    <element name="sourceurl" maxlength="200" card="1:1"/>
   </element>
   <element name="interestingfact" maxlength="" card="1:1">
    <element name="url" maxlength="200" card="1:1"/>
    <element name="text" maxlength="523" card="1:1"/>
   </element>
   <element name="subjects" maxlength="" card="1:1">
    <element name="subject" maxlength="" card="1:n(30)">
     <element name="category" maxlength="27" card="1:1"/>
     <element name="subcategory" maxlength="57" card="1:1"/>
     <element name="primary" maxlength="3" card="1:1"/>
    </element>
   </element>
   <element name="resourcekeywords" maxlength="" card="0:1">
    <element name="keywords" maxlength="" card="1:1">
     <element name="term" maxlength="40" card="1:n(19)"/>
    </element>
   </element>
  </element>
 </element>
</elements>
\end{BVerbatim}

\end{appendix}


\end{document}

