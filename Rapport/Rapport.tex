%% Based on a TeXnicCenter-Template by Tino Weinkauf.
%%%%%%%%%%%%%%%%%%%%%%%%%%%%%%%%%%%%%%%%%%%%%%%%%%%%%%%%%%%%%

%%%%%%%%%%%%%%%%%%%%%%%%%%%%%%%%%%%%%%%%%%%%%%%%%%%%%%%%%%%%%
%% HEADER
%%%%%%%%%%%%%%%%%%%%%%%%%%%%%%%%%%%%%%%%%%%%%%%%%%%%%%%%%%%%%
\documentclass[a4paper,twoside,10pt]{report}
% Alternative Options:
%	Paper Size: a4paper / a5paper / b5paper / letterpaper / legalpaper / executivepaper
% Duplex: oneside / twoside
% Base Font Size: 10pt / 11pt / 12pt


%% Language %%%%%%%%%%%%%%%%%%%%%%%%%%%%%%%%%%%%%%%%%%%%%%%%%
\usepackage[Danish]{babel} %francais, polish, spanish, ...
\usepackage[T1]{fontenc}
\usepackage[ansinew]{inputenc}

%\usepackage{lmodern} %Type1-font for non-english texts and characters


%% Packages for Graphics & Figures %%%%%%%%%%%%%%%%%%%%%%%%%%
%\usepackage{graphicx} %%For loading graphic files
%\usepackage{subfig} %%Subfigures inside a figure

%% Please note:
%% Images can be included using \includegraphics{Dateiname}
%% resp. using the dialog in the Insert menu.
%% 
%% The mode "LaTeX => PDF" allows the following formats:
%%   .jpg  .png  .pdf  .mps
%% 
%% The modes "LaTeX => DVI", "LaTeX => PS" und "LaTeX => PS => PDF"
%% allow the following formats:
%%   .eps  .ps  .bmp  .pict  .pntg


%% Math Packages %%%%%%%%%%%%%%%%%%%%%%%%%%%%%%%%%%%%%%%%%%%%
%\usepackage{amsmath}
%\usepackage{amsthm}
%\usepackage{amsfonts}


%% Line Spacing %%%%%%%%%%%%%%%%%%%%%%%%%%%%%%%%%%%%%%%%%%%%%
%\usepackage{setspace}
%\singlespacing        %% 1-spacing (default)
%\onehalfspacing       %% 1,5-spacing
%\doublespacing        %% 2-spacing


%% Other Packages %%%%%%%%%%%%%%%%%%%%%%%%%%%%%%%%%%%%%%%%%%%
\usepackage{a4wide} %%Smaller margins = more text per page.
\usepackage{fancyhdr} %%Fancy headings
\usepackage{hyperref} %make TOC linkable
%\usepackage{longtable} %%For tables, that exceed one page


%%%%%%%%%%%%%%%%%%%%%%%%%%%%%%%%%%%%%%%%%%%%%%%%%%%%%%%%%%%%%
%% Options / Modifications
%%%%%%%%%%%%%%%%%%%%%%%%%%%%%%%%%%%%%%%%%%%%%%%%%%%%%%%%%%%%%
%% Based on a TeXnicCenter-Template by Tino Weinkauf.
%%%%%%%%%%%%%%%%%%%%%%%%%%%%%%%%%%%%%%%%%%%%%%%%%%%%%%%%%%%%%

%%%%%%%%%%%%%%%%%%%%%%%%%%%%%%%%%%%%%%%%%%%%%%%%%%%%%%%%%%%%%
%% OPTIONS
%%%%%%%%%%%%%%%%%%%%%%%%%%%%%%%%%%%%%%%%%%%%%%%%%%%%%%%%%%%%%
%%
%% ATTENTION: You need a main file to use this one here.
%%            Use the command "\input{filename}" in your
%%            main file to include this file.
%%

%%%%%%%%%%%%%%%%%%%%%%%%%%%%%%%%%%%%%%%%%%%%%%%%%%%%%%%%%%%%%
%% OPTIONS FOR ITEMIZE
%%%%%%%%%%%%%%%%%%%%%%%%%%%%%%%%%%%%%%%%%%%%%%%%%%%%%%%%%%%%%
\renewcommand{\labelitemii}{$\circ$} % angiver typografien imtemize - niveau 2 til at være en tom cirkel

%%%%%%%%%%%%%%%%%%%%%%%%%%%%%%%%%%%%%%%%%%%%%%%%%%%%%%%%%%%%%
%% NEW COMMANDS FOR FORMATTING
%%%%%%%%%%%%%%%%%%%%%%%%%%%%%%%%%%%%%%%%%%%%%%%%%%%%%%%%%%%%%
\newcommand{\fil}[1]{#1} %formatering for filnavne
\newcommand{\prg}[1]{\textsc{#1}} %formatering for programnavne
\newcommand{\lang}[1]{#1} %formatering for programmeringssprog
\newcommand{\funk}[1]{\textbf{#1}} %formatering for funktionsnavne
\newcommand{\var}[1]{\textbf{\$#1}} %formatering for variabel/parameter navne
\newcommand{\lref}[1]{\texttt{#1}} %formatering for kodeeksempel linienumre
\newcommand{\xmlatt}[1]{\texttt{#1}} %formatering for xml element atributter
\newcommand{\xmlelm}[1]{\texttt{#1}} %formatering for xml element navne

%%%%%%%%%%%%%%%%%%%%%%%%%%%%%%%%%%%%%%%%%%%%%%%%%%%%%%%%%%%%%
%% OPTIONS FOR FANCYHEADER
%%%%%%%%%%%%%%%%%%%%%%%%%%%%%%%%%%%%%%%%%%%%%%%%%%%%%%%%%%%%%
\pagestyle{fancyplain} %til at lave overskrift på hver side
%\setlength{\parindent}{0pt} %indrykning ved ny sektion
\fancyhf{} % delete current header and footer
\fancyhead[OL]{\leftmark}
\fancyhead[OR]{\thepage}
\fancyhead[EL]{\thepage}
\fancyhead[ER]{\leftmark} %giver overskrift \rightmark giver subsection overskrift
\fancypagestyle{plain}{%
\fancyhead{} % get rid of headers on plain pages
\renewcommand{\headrulewidth}{0pt} % and the line
}



%%%%%%%%%%%%%%%%%%%%%%%%%%%%%%%%%%%%%%%%%%%%%%%%%%%%%%%%%%%%%
%% OPTIONS FOR HYPERREF
%%%%%%%%%%%%%%%%%%%%%%%%%%%%%%%%%%%%%%%%%%%%%%%%%%%%%%%%%%%%%
\hypersetup{
    colorlinks,
    citecolor=black,
    filecolor=black,
    linkcolor=black,
    urlcolor=black
}


%%Space between paragraphs: half the height of the small x
%\setlength{\parskip}{0.5ex}

%%Indent at the beginning of a paragraph: set to zero
%\setlength{\parindent}{0ex}

%%Spacing between lines: 1.5 times
%% ==> Consider using the package 'setspace' instead.
%\linespread{1.5}


 %You need a file 'options.tex' for this


%%%%%%%%%%%%%%%%%%%%%%%%%%%%%%%%%%%%%%%%%%%%%%%%%%%%%%%%%%%%%
%% DOCUMENT
%%%%%%%%%%%%%%%%%%%%%%%%%%%%%%%%%%%%%%%%%%%%%%%%%%%%%%%%%%%%%
\begin{document}

\pagestyle{empty} %No headings for the first pages.


%% Title Page %%%%%%%%%%%%%%%%%%%%%%%%%%%%%%%%%%%%%%%%%%%%%%%
%% ==> Write your text here or include other files.

%% The simple version:
\title{Xml i databaser}
\author{Andreas Falk}
\date{28-01-2014} %%If commented, the current date is used.
\maketitle

%% The nice version:
%%% Based on a TeXnicCenter-Template by Tino Weinkauf.
%%%%%%%%%%%%%%%%%%%%%%%%%%%%%%%%%%%%%%%%%%%%%%%%%%%%%%%%%%%%%

%%%%%%%%%%%%%%%%%%%%%%%%%%%%%%%%%%%%%%%%%%%%%%%%%%%%%%%%%%%%%
%% Deckblatt
%%%%%%%%%%%%%%%%%%%%%%%%%%%%%%%%%%%%%%%%%%%%%%%%%%%%%%%%%%%%%
%%
%% ATTENTION: You need a main file to use this one here.
%%            Use the command "\input{filename}" in your
%%            main file to include this file.
%%

%\begin{titlepage}

\begin{center}

%\vspace*{1cm}
\Large
\textsc{XPath, XQuery og SQLXML}\\

\vspace{5cm}

%\LARGE
\textsc{Databaser for udviklere: kompleks data og logik i databasen\\[0.5\baselineskip]
af\\[0.5\baselineskip]
Andreas Falk\\
{\normalsize \textsc{}}}\\

\vspace{5cm}
\textsc{\28-01-2014}\\ %%Date - better you write it yourself.

\vspace{1cm}
\textsc{%Supervisors:\\
%Prof. V. Nachname\\
%Dr. V. Nachname
}\\

\vspace{1cm}
\textsc{AAU\\
%Faculty of ...\\
%Institute of ...
}\\

\end{center}

%\end{titlepage}
 %%You need a file 'titlepage.tex' for this.
%% ==> TeXnicCenter supplies a possible titlepage file
%% ==> with its templates (File | New from Template...).


%% Inhaltsverzeichnis %%%%%%%%%%%%%%%%%%%%%%%%%%%%%%%%%%%%%%%
\tableofcontents %Table of contents
\cleardoublepage %The first chapter should start on an odd page.

\pagestyle{plain} %Now display headings: headings / fancy / ...


%% Chapters %%%%%%%%%%%%%%%%%%%%%%%%%%%%%%%%%%%%%%%%%%%%%%%%%
\chapter{Introduktion}
\label{chap:Introduktion}
\section{Form�l}
\label{sec:Formaal}
Denne rapport handler prim�rt om hvordan xml dokumenter kan valideres, uds�ttes for effektive foresp�rgelser og sameksistere med og i en SQL database. Alle emner bliver behandlet ved brug af et gennemg�ende eksempel, som l�bende uds�ttes for alle de forskellige teknologier som n�vnes herunder.
\begin{itemize}
	\item XML Skema
	\item DTD
	\item XPath
	\item XQuery
	\item SQL/XML
	\item Full text search
\end{itemize}

\subsection{Fremgangsm�de}
\label{subsec:Fremgangsmaade}
Da det er generelt er umuligt at behandle et XML dokument, hvis ikke strukturen er kendt, skal der laves en mindre analyse af dette. Det er hvad kapitel \ref{chap:analyse_xml_struktur} omhandler, hvor XQuery og XPath benyttes. I kapitel \ref{chap:XML_skema_for_resource} udnyttes det at strukturen for xml dokumentet nu er defineret, og der oprettes et XML skema for at dokumentere dette p� en standard m�de, og dermed ogs� muligg�re validering af dokumentet.  Kapitel \ref{chap:XML_i_SQL_DB} handler om hvordan XMl dokumenter kan indl�ses i en database, p� forskellige m�der, samt at data kan udtr�kkes fra en database, og leveres som XML direkte fra databasen, uden brug af ekstra applikationer.  *TODO* beskriv resten af kapitlerne

Men f�r der kan startes p� alt dette, kommer f�rst en beskrivelse af det xml dokument, som rapporten benytter som datakilde.

\subsection{XML dokumentet resource}
\label{subsec:XML_dokumentet_resource}
XML dokumentet er gemt i filen \fil{resource.xml}, som er en kopi af filen \fil{gemexport.xml} fra kursets hjemmeside. Filen er at finde i mappen Data i projektmappen. Omd�bningen skyldes at navnet gemexport ikke var sigende for indholdet. 

Dataene i XML dokumentet er en liste over uddannelsesresourcer, hvor der for hver resource er en m�ngde oplysninger. Der er ud over titel og beskrivelse ogs� oplysninger om udgiver, kategorier og hvorn�r resourcen er oprettet og meget andet.  

I resten af rapporten vil brugen af navnet resource b�de referere til XML dokumentet, som er gemt i filen resource.xml, og til den instans af BaseX databasen der oprettes ud fra filen resource.xml. I tilf�lde hvor betydningen ikke fremg�r af sammenh�ngen, vil betydningen blive pr�ciseret eksplicit.

%\begin{itemize}
%	\item Implementation af effektiv h�ndtering af xml-filen gemexport.xml.
%	\begin{itemize}
%		\item F�rste skridt er en analyse af strukturen, hvor resultatet er et xml skema, som kan validere vores xml fil. Dette g�res ved brug af BaseX,  hvor der oprettes en ny database p� grundlag af resource.xml.
%		\item Flere forskellige foresp�rgelser skrives i hhv. Xpath og Xquery,  som bl.a. viser kardinalitetten af  elementer i xml'en. Samt om elementer er oprettet, men er tomme elementer. 
%		\item Som udgangspunkt antages det at xml'en indeholder alle yderligheder der kan forekomme, s� f.eks. maks l�ngde for et tekst element kan udledes vha. Foresp�rgelser mod xml'en.
%		\item Der inds�ttes en reference til det nyoprettede skema, i den oprindelige xml, og skemaet valideres med xml plugin i Notepad++. Dette kan selvf�lgelig g�res med et hvilken som helst xml valideringsv�rkt�j.

%	\end{itemize}
%	\item Nu skal xml'en gemmes i en relationel database, og her kan der v�lges flere tilgangsm�der.
%	\begin{itemize}
%		\item At modellere xml strukturen vha. Den Entitets og relationelle model, og splitte(shredde) xml op og inds�tte den, s� vi p� den m�de kan tilg� data og manipulere med den som en hvilken som helst anden sql database.
%		\item vi kan v�lge at inds�tte hele vores xmlfil i en r�kke i en tabel, og dern�st prim�rt benytte Xpath foresp�rgelser til foresp�rgelser. Disse foresp�rgelser kan, og vil sikkert blive suppleret med brug af XmlSql funktioner for at strukturere resultatet af Xpath foresp�rgelserne yderligere.
%		\item Sidste mulighed er delvis shredding, hvor f.eks.  de enkelte resourcer splittes ud i enkelte r�kker, med enkelte vigtige attributer, id, evt titel, trulket ud i egne kolonner, for at simplificere filtrering p� disse attributter.
%	\end{itemize}
%	\item Indl�sning foreg�r i postgresql 9.3, som kun underst�tter Xpath 1.0, dvs. Ingen Xquery og enkelte funktioner fra nutidig Xpath kan ikke benyttes. Derudover mangler postgresql underst�ttelse af xmlsql funktionen xmltable,  som ellers er oplagt til mapping fra xml til relationelle tabeller. Hvorimod de implementerede publishing functions er implementeret ok.
%	\end{itemize}







 
\chapter{XML skema for resource}
\label{chap:XML_skema_for_resource}
\section{Form�l}
\label{sec:formaal_skemaet}
Det prim�re form�l med et xml skema, er muligheden for at validere et xml dokument, og p� den m�de kontrollere at det aftalte format er overholdt. Ofte kan disse skemaer, som i sig selv er xml dokumenter, ogs� bruges til at automatisk generering af kode til behandling af alle de xml dokumenter som er valide i forhold til skemaet. Dette sker typisk gennem tredjeparts software. 
Et xml skema skal beskrive 2 vigtige egenskaber for at opfylde det beskrevne behov. 
\begin{description}
	\item[Struktur] som beskriver forholdet mellem for�ldre og b�rn elementer. Her kommer complexType og sequence med flere i spil
	\item[Datatype] som angiver format eller v�rdis�t for det data der er gemt i elementerne. Her benyttes simpleType og restriction med flere.
\end{description}
Med analyse funktionen fra kapitel \ref{chap:analyse_xml_struktur}, er det nemt at komme igang med at definere et skema, som kan validere strukturen i xml dokumentet. Hvor godt resultatet fra analyse funktionen end er, s� er fokus for den prim�rt elementerne og deres indbyrdes forhold til hinanden. S� med den information er grundstrukturen i skemaet klarlagt.
Men resultatet indeholder ikke nogen information om datatyperne, der er sv�rt at fastsl� automatisk, da det typisk foruds�tter en del dom�ne viden. Datatyperne vil derfor i h�j grad blive udledt udfra deres navne og implicitte betydning. Nogle af begr�nsningerne der benyttes i skemaet, er tilf�jet udelukkende for at demonstrere flere forskellige typer af begr�nsninger, og vil m�ske ikke v�re korrekte i XML dokumentets virkelige dom�ne.

\section{Skemaet \fil{resource.xsd}}
\label{sec:skemaet}

Det endelige skema kan ses i filen \fil{resource.xsd}, samt i bilag XXX *TODO* reference til bilag med hele skemaet *TODO*. Skemaet er opbygget, ved en manuel gennemgang af analyse resultatet i bilag *TODO reference til bilaget*. Hvert element er behandlet, og en type med en passende defintion er oprettet i skemaet. Der er brugt navngivne typer i stor udstr�kning, da det �ger l�sbarheden v�sentligt. 
Elementerne i skemaet beskrives her, grupperet udfra hvordan de er defineret. 
\vspace{1.5mm}
\begin{compactitem}
  \item 
F�lgende elementer i dokumentet er defineret som komplekse, med en sekvens af underelementer:
  \begin{compactitem}
    \item \xmlelm{root} anonym type
		\item \xmlelm{resource} resourceType
		\item \xmlelm{itemdate} itemdateType
		\item \xmlelm{identifier} identifierType
		\item \xmlelm{publisher} publisherType
		\item \xmlelm{image} imageType
		\item \xmlelm{interestingfact} interestingfactType
		\item \xmlelm{subjects} subjectsType
		\item \xmlelm{subject} subejctType
		\item \xmlelm{resourcekeywords} resourcekeywordsType
		\item \xmlelm{keywords} anonym type
  \end{compactitem}
	\vspace{1.0mm}
G�ldende for en sekvens er at alle elementerne skal optr�de i den angivede r�kkef�lge. Dette er i mods�tning til en anden indikator all, hvor om det g�lder at elementerne kan optr�de i en vilk�rlig r�kkef�lge. 
For hvert element i sekvensen kan der angives kardinalitet, ved brug af indikatorerne minOccurs og maxOccurs, og her er brugt de v�rdier som analyse funktionen returnerede.
	
	\vspace{1.0mm}
	\item Disse elementer er defineret som simple typer, med en begr�nsing angivet i forhold til den valgte basis type:
	\begin{compactitem}
	  \item \xmlelm{title} titleType oprettes med basistypen string, og minLength og maxLength angives til hhv. 5 og 110. 
		\item \xmlelm{url} urlType oprettes med basistypen string, og med et meget simpelt pattern der matcher formattet for en url.
		\item \xmlelm{sourceurl} refererer til den navngivne type urlType, der er forklaret ovenfor.
	  \item \xmlelm{primary} primaryType oprettes med basistypen string, og en enumeration der angiver de valide v�rdier for elementet.
	\end{compactitem}
	\vspace{1.0mm}	
  Disse typer illustrerer meget godt brugen af begr�nsninger, hvor det i forvejen udbyggede typehieraki nemt kan udvides med mere specialiserede typer.
	
	\vspace{1.0mm}
  \item Disse elementer defineres ved brug af standard typer:
	\begin{compactitem}
 		\item \xmlelm{ID} positivInteger
    \item \xmlelm{description} string
		\item \xmlelm{category} string
		\item \xmlelm{subcategory} string
		\item \xmlelm{term} string
  	\item \xmlelm{text} string
		\item \xmlelm{caption} string
		\item \xmlelm{alttext} string
		\item \xmlelm{name} string
		\item \xmlelm{agency} string
		\item \xmlelm{recordcreated} date
		\item \xmlelm{placedonline} date
  \end{compactitem}
\end{compactitem}
\vspace{1.5mm}

\section{Konklusion}
\label{sec:Konklusion_xmlskema}
Med et skema defineret, kan xml strukturen valideres. Det g�res ved at tilf�je en reference til skemaet i rod elementet. I dette projekt er \prg{Notepad++} samt pluginet 'xml tools' brugt til at foretage valideringen. 

\begin{figure}[ht]
\centering
\begin{BVerbatim} 
<root xmlns:xsi="http://www.w3.org/2001/XMLSchema-instance" 
      xsi:noNamespaceSchemaLocation="resource.xsd">
\end{BVerbatim}
\caption{Reference til skema tilf�jet i dokumentroden}
\label{code:reference_til_skema}
\end{figure}

\noindent Ikke overaskende er strukturen valid i forhold til det nye skema.



%%%%%%%%%%%%%%%%%%%%%%%%%%%%%%%%%%%%%%%%%%%%%%%%%%%%%%%%%%%%%
%% BIBLIOGRAPHY AND OTHER LISTS
%%%%%%%%%%%%%%%%%%%%%%%%%%%%%%%%%%%%%%%%%%%%%%%%%%%%%%%%%%%%%
%% A small distance to the other stuff in the table of contents (toc)
%\addtocontents{toc}{\protect\vspace*{\baselineskip}}

%% The Bibliography
%% ==> You need a file 'literature.bib' for this.
%% ==> You need to run BibTeX for this (Project | Properties... | Uses BibTeX)
%\addcontentsline{toc}{chapter}{Bibliography} %'Bibliography' into toc
%\nocite{*} %Even non-cited BibTeX-Entries will be shown.
%\bibliographystyle{alpha} %Style of Bibliography: plain / apalike / amsalpha / ...
%\bibliography{literature} %You need a file 'literature.bib' for this.

%% The List of Figures
%\clearpage
%\addcontentsline{toc}{chapter}{List of Figures}
%\listoffigures

%% The List of Tables
%\clearpage
%\addcontentsline{toc}{chapter}{List of Tables}
%\listoftables


%%%%%%%%%%%%%%%%%%%%%%%%%%%%%%%%%%%%%%%%%%%%%%%%%%%%%%%%%%%%%
%% APPENDICES
%%%%%%%%%%%%%%%%%%%%%%%%%%%%%%%%%%%%%%%%%%%%%%%%%%%%%%%%%%%%%
%\appendix
%% ==> Write your text here or include other files.

%\input{FileName} %You need a file 'FileName.tex' for this.


\end{document}

