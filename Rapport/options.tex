%% Based on a TeXnicCenter-Template by Tino Weinkauf.
%%%%%%%%%%%%%%%%%%%%%%%%%%%%%%%%%%%%%%%%%%%%%%%%%%%%%%%%%%%%%

%%%%%%%%%%%%%%%%%%%%%%%%%%%%%%%%%%%%%%%%%%%%%%%%%%%%%%%%%%%%%
%% OPTIONS
%%%%%%%%%%%%%%%%%%%%%%%%%%%%%%%%%%%%%%%%%%%%%%%%%%%%%%%%%%%%%
%%
%% ATTENTION: You need a main file to use this one here.
%%            Use the command "\input{filename}" in your
%%            main file to include this file.
%%

%%%%%%%%%%%%%%%%%%%%%%%%%%%%%%%%%%%%%%%%%%%%%%%%%%%%%%%%%%%%%
%% OPTIONS FOR ITEMIZE
%%%%%%%%%%%%%%%%%%%%%%%%%%%%%%%%%%%%%%%%%%%%%%%%%%%%%%%%%%%%%
\renewcommand{\labelitemii}{$\circ$} % angiver typografien imtemize - niveau 2 til at være en tom cirkel

%%%%%%%%%%%%%%%%%%%%%%%%%%%%%%%%%%%%%%%%%%%%%%%%%%%%%%%%%%%%%
%% NEW COMMANDS FOR FORMATTING
%%%%%%%%%%%%%%%%%%%%%%%%%%%%%%%%%%%%%%%%%%%%%%%%%%%%%%%%%%%%%
\newcommand{\fil}[1]{#1} %formatering for filnavne
\newcommand{\prg}[1]{\textsc{#1}} %formatering for programnavne
\newcommand{\lang}[1]{#1} %formatering for programmeringssprog
\newcommand{\funk}[1]{\textbf{#1}} %formatering for funktionsnavne
\newcommand{\var}[1]{\textbf{\$#1}} %formatering for variabel/parameter navne
\newcommand{\lref}[1]{\texttt{#1}} %formatering for kodeeksempel linienumre
\newcommand{\xmlatt}[1]{\texttt{#1}} %formatering for xml element atributter
\newcommand{\xmlelm}[1]{\texttt{#1}} %formatering for xml element navne

%%%%%%%%%%%%%%%%%%%%%%%%%%%%%%%%%%%%%%%%%%%%%%%%%%%%%%%%%%%%%
%% OPTIONS FOR FANCYHEADER
%%%%%%%%%%%%%%%%%%%%%%%%%%%%%%%%%%%%%%%%%%%%%%%%%%%%%%%%%%%%%
\pagestyle{fancyplain} %til at lave overskrift på hver side
%\setlength{\parindent}{0pt} %indrykning ved ny sektion
\fancyhf{} % delete current header and footer
\fancyhead[OL]{\leftmark}
\fancyhead[OR]{\thepage}
\fancyhead[EL]{\thepage}
\fancyhead[ER]{\leftmark} %giver overskrift \rightmark giver subsection overskrift
\fancypagestyle{plain}{%
\fancyhead{} % get rid of headers on plain pages
\renewcommand{\headrulewidth}{0pt} % and the line
}



%%%%%%%%%%%%%%%%%%%%%%%%%%%%%%%%%%%%%%%%%%%%%%%%%%%%%%%%%%%%%
%% OPTIONS FOR HYPERREF
%%%%%%%%%%%%%%%%%%%%%%%%%%%%%%%%%%%%%%%%%%%%%%%%%%%%%%%%%%%%%
\hypersetup{
    colorlinks,
    citecolor=black,
    filecolor=black,
    linkcolor=black,
    urlcolor=black
}


%%Space between paragraphs: half the height of the small x
%\setlength{\parskip}{0.5ex}

%%Indent at the beginning of a paragraph: set to zero
%\setlength{\parindent}{0ex}

%%Spacing between lines: 1.5 times
%% ==> Consider using the package 'setspace' instead.
%\linespread{1.5}


