\chapter{XML skema for resource}
\label{chap:XML_skema_for_resource}
\section{Form�l}
\label{sec:formaal_skemaet}
Det prim�re form�l med et XML skema, er muligheden for at validere et XML dokument, og p� den m�de kontrollere at det aftalte format er overholdt. Ofte kan disse skemaer, som i sig selv er XML dokumenter, ogs� bruges til at automatisk generering af kode til behandling af alle de XML dokumenter som er valide i forhold til skemaet. Dette sker typisk gennem tredjeparts software. 
Et XML skema skal beskrive 2 vigtige egenskaber for at opfylde det beskrevne behov. 
\begin{description}
	\item[Struktur] som beskriver forholdet mellem for�ldre og b�rn elementer. Her kommer complexType og sequence med flere i spil
	\item[Datatype] som angiver format eller v�rdis�t for det data der er gemt i elementerne. Her benyttes simpleType og restriction med flere.
\end{description}
Med analyse funktionen fra kapitel \ref{chap:analyse_xml_struktur}, er det nemt at komme igang med at definere et skema, som kan validere strukturen i xml dokumentet. S� med resultatet fra den automatiske analyse er grundstrukturen i skemaet klarlagt.
Men resultatet fra \funk{analyze} indeholder ikke nogen information om datatyperne. Det kan v�re sv�rt at fastsl� disse automatisk, da det typisk foruds�tter en del dom�ne viden. Datatyperne vil derfor i h�j grad blive udledt udfra deres navne og manuel genkendelse af f.eks. et dato format. Nogle af begr�nsningerne(restrictions) der benyttes i skemaet, er tilf�jet udelukkende for at demonstrere flere forskellige typer af begr�nsninger, og vil m�ske ikke v�re korrekte i XML dokumentets virkelige dom�ne, p� trods af at denne instans af dokumentet overholder dem.

\section{Skemaet \fil{resource.xsd}}
\label{sec:skemaet}

Det endelige skema kan ses i filen \fil{Data/resource.xsd}. Skemaet er opbygget, ved en manuel gennemgang af analyse resultatet i bilag \ref{app:Genereret analyse af resource.xml}. Hvert element er behandlet, og en type med en passende defintion er oprettet i skemaet. Der er brugt navngivne typer i stor udstr�kning, da det �ger l�sbarheden v�sentligt. 
Elementerne i skemaet beskrives her, grupperet udfra hvordan de er defineret. 
\vspace{1.5mm}
\begin{compactitem}
  \item 
F�lgende elementer i dokumentet er defineret som komplekse, med en sekvens af underelementer:
  \begin{compactitem}
    \item \xmlelm{root} anonym type
		\item \xmlelm{resource} resourceType
		\item \xmlelm{itemdate} itemdateType
		\item \xmlelm{identifier} identifierType
		\item \xmlelm{publisher} publisherType
		\item \xmlelm{image} imageType
		\item \xmlelm{interestingfact} interestingfactType
		\item \xmlelm{subjects} subjectsType
		\item \xmlelm{subject} subejctType
		\item \xmlelm{resourcekeywords} resourcekeywordsType
		\item \xmlelm{keywords} anonym type
  \end{compactitem}
	\vspace{1.0mm}
G�ldende for en sekvens er at alle elementerne skal optr�de i den angivede r�kkef�lge. Dette er i mods�tning til en anden indikator all, hvor om det g�lder at elementerne kan optr�de i en vilk�rlig r�kkef�lge. 
For hvert element i sekvensen kan der angives kardinalitet, ved brug af indikatorerne minOccurs og maxOccurs, og her er brugt de v�rdier som analyse funktionen returnerede.
	
	\vspace{1.0mm}
	\item Disse elementer er defineret som simple typer, med en begr�nsing angivet i forhold til den valgte basis type:
	\begin{compactitem}
	  \item \xmlelm{title} titleType oprettes med basistypen string, og minLength og maxLength angives til hhv. 5 og 110. 
		\item \xmlelm{url} urlType oprettes med basistypen string, og med et meget simpelt pattern der matcher formattet for en url.
		\item \xmlelm{sourceurl} refererer til den navngivne type urlType, der er forklaret ovenfor.
	  \item \xmlelm{primary} primaryType oprettes med basistypen string, og en enumeration der angiver de valide v�rdier for elementet.
	\end{compactitem}
	\vspace{1.0mm}	
  Disse typer illustrerer meget godt brugen af begr�nsninger, hvor det i forvejen udbyggede typehieraki nemt kan udvides med mere specialiserede typer.
	
	\vspace{1.0mm}
  \item Disse elementer defineres ved brug af standard typer:
	\begin{compactitem}
 		\item \xmlelm{ID} positivInteger
    \item \xmlelm{description} string
		\item \xmlelm{category} string
		\item \xmlelm{subcategory} string
		\item \xmlelm{term} string
  	\item \xmlelm{text} string
		\item \xmlelm{caption} string
		\item \xmlelm{alttext} string
		\item \xmlelm{name} string
		\item \xmlelm{agency} string
		\item \xmlelm{recordcreated} date
		\item \xmlelm{placedonline} date
  \end{compactitem}
\end{compactitem}
\vspace{1.5mm}

\section{Konklusion}
\label{sec:Konklusion_xmlskema}
Med et skema defineret, kan xml strukturen valideres. Det g�res ved at tilf�je en reference til skemaet i rod elementet. Dette er gjort ved at lave en kopi af resource.xml, kaldet resource\_with\_namespace.xml, og s� tilf�je koden fra figur \ref{code:reference_til_skema} i den fil. Dette er gjort, for ikke at redigere i kilde-filen. I dette projekt er \prg{Notepad++} samt pluginet 'xml tools' brugt til at foretage valideringen. 

\begin{figure}[ht]
\centering
\begin{BVerbatim} 
<root xmlns:xsi="http://www.w3.org/2001/XMLSchema-instance" 
      xsi:noNamespaceSchemaLocation="resource.xsd">
\end{BVerbatim}
\caption{Reference til skema tilf�jet i dokumentroden}
\label{code:reference_til_skema}
\end{figure}

\noindent Ikke overaskende er strukturen valid i forhold til det nye skema.

