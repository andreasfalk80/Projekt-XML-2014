\chapter{XML skema for resource.xml}
\label{sec:XML skema}
\section{Form�l}
\label{sec:formaal_skemaet}
Det prim�re form�l med et xml skema, er muligheden for at validere et xml dokument, og p� den m�de kontrollere at det aftalte format er overholdt. Ofte kan disse skemaer, som i sig selv er xml dokumenter, ogs� bruges til at automatisk generering af kode til behandling af alle de xml dokumenter som er valide i forhold til skemaet. Dette sker typisk gennem tredjeparts software. 

Et xml skema skal beskrive 2 vigtige egenskaber for at opfylde det beskrevne behov. 
\begin{description}
	\item[Strukturen] som beskriver forholdet mellem for�ldre og b�rn elementer. Her kommer complexType, sequence med flere i spil
	\item[Datatyper] som angiver format eller v�rdis�t for det data der er gemt i elementerne. Her benyttes simpleType, restriction og enumeration med flere.
\end{description}

Med den tidligere beskrevet analyse funktion, som giver resultatet der kan ses i figur \ref{code:xml_analyse_result}, er det relativt nemt at definere et skema, som kan validere xml strukturen.
 Men ud over informationen om den maksimale l�ngde for et tekst element, indeholder resultatet ikke nogen information om datatyperne. Disse vil derfor i h�j grad blive udledt udfra deres navne og implicitte betydning. Nogle af begr�nsningerne der kan ses i skemaet, er tilf�jet udelukkende for at demonstrere flere forskellige typer af begr�nsninger.

\section{Skemaet}
\label{sec:skemaet}
Elementerne i skemaet beskrives her, grupperet udfra hvilke typer og skema 'features' der er benyttet til at definere dem. 

\begin{description}
\item[complexType og sequence] Typen resourceType er et eksempel p� en type der er defineret som en kompleks type, der best�r af en sekvens af elementer. 
G�ldende for en sekvens er at alle elementerne skal optr�de i den angivede r�kkef�lge. Dette er i mods�tning til en anden indikator, all, hvor om det g�lder at elementerne kan optr�de i en vilk�rlig r�kkef�lge. 
For hvert element i sekvensen kan der angives kardinalitet, ved brug af indikatorerne minOccurs og maxOccurs. Her er brugt de v�rdier som analyse funktionen returnerede.

Alle definerede typer, p�n�r titleType, urlType og primaryType er komplekse typer med en sekvens. Nogle er lavet som navngivede typer, f.eks. resourceType. Mens andre er anonyme typer, som for eksempel typen for elementet keywords, som er defineret i den navngivne type resourcekeywordsType.
%Der m� ikke rettes i formattering, s� g�r det galt. Kode b�r kopieres over noteblok, for at fjerne ekstra formateringen fra Notepad++
\begin{figure}[ht]
\centering
\begin{BVerbatim} 
<xs:complexType name='resourcekeywordsType'>
  <xs:sequence>
    <xs:element name='keywords'>
      <xs:complexType>
        <xs:sequence>
	  <xs:element name='term' 
                      type='xs:string' 
                      maxOccurs='unbounded'/>
        </xs:sequence>
      </xs:complexType>
    </xs:element>
  </xs:sequence>
</xs:complexType>
\end{BVerbatim}
\caption{Eksempel p� navngivet og anonym type, der samtidig er kompleks}
\label{code:def_resourcekeywordsType}
\end{figure}

\item[simpleType og restriction] Typen titleType tager udgangspunkt i typen string, men tilf�jer en begr�nsning p� l�ngen af strengen, ved brug af minLength og maxLength. UrlType bruger facetten pattern, til at angive et simpelt format tjek af en url adresse, og den sidste simple type, primaryType bruger facetten enumeration, til at begr�nse v�rdierne til YES og NO.
%Der m� ikke rettes i formattering, s� g�r det galt. Kode b�r kopieres over noteblok, for at fjerne ekstra formateringen fra Notepad++
\begin{figure}[ht]
\centering
\begin{BVerbatim} 
<xs:simpleType name='urlType'>
  <xs:restriction base='xs:string'>
    <xs:pattern value="((http|https)?://.*)?" />
  </xs:restriction>
</xs:simpleType>
\end{BVerbatim}
\caption{Eksempel simpel type, med en begr�nsning som anvender et pattern}
\label{code:def_urlType}
\end{figure}

\item[standard typer] Enkelte elementer benytter sig af standard typer, f.eks. ID som defineres som positiveInteger, og description som en string uden begr�nsninger, samt recordcreated og placedonline i itemDate, der begge bruger typen date. 
\end{description}


*TODO* ret til s� sequence bliver refereret som en indikator *TODO*


*TODO* reference til bilag med hele skemaet *TODO*

%Skal placeres i bilag
\begin{figure}[ht]
\centering
\begin{BVerbatim}
<elements>
  <element name="root" maxlength="" card="1:1">
    <element name="resource" maxlength="" card="1:n(1741)">
      <element name="ID" maxlength="4" card="1:1"/>
      <element name="title" maxlength="107" card="1:1"/>
      <element name="description" maxlength="500" card="1:1"/>
      <element name="itemdate" maxlength="" card="1:1">
        <element name="recordcreated" maxlength="10" card="1:1"/>
        <element name="placedonline" maxlength="10" card="1:1"/>
      </element>
      <element name="identifier" maxlength="" card="1:1">
        <element name="url" maxlength="195" card="1:1"/>
      </element>
      <element name="publisher" maxlength="" card="1:1">
        <element name="name" maxlength="68" card="1:1"/>
        <element name="agency" maxlength="58" card="1:1"/>
      </element>
      <element name="image" maxlength="" card="1:1">
        <element name="url" maxlength="71" card="1:1"/>
        <element name="caption" maxlength="95" card="1:1"/>
        <element name="alttext" maxlength="200" card="1:1"/>
        <element name="sourceurl" maxlength="200" card="1:1"/>
      </element>
      <element name="interestingfact" maxlength="" card="1:1">
        <element name="url" maxlength="200" card="1:1"/>
        <element name="text" maxlength="523" card="1:1"/>
      </element>
      <element name="subjects" maxlength="" card="1:1">
        <element name="subject" maxlength="" card="1:n(30)">
          <element name="category" maxlength="27" card="1:1"/>
          <element name="subcategory" maxlength="57" card="1:1"/>
          <element name="primary" maxlength="3" card="1:1"/>
        </element>
      </element>
      <element name="resourcekeywords" maxlength="" card="0:1">
        <element name="keywords" maxlength="" card="1:1">
          <element name="term" maxlength="40" card="1:n(19)"/>
        </element>
      </element>
    </element>
  </element>
</elements>
\end{BVerbatim}
\caption{Resultat fra k�rsel af xml analyse}
\label{code:xml_analyse_result}
\end{figure}
