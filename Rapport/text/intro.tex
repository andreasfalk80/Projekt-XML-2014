\chapter{Introduktion}
\label{sec:Introduktion}


\section{Form�l}
\label{sec:Form�l}
Beskrivelse af xml filen og dens data der skal behandles \ldots

\begin{itemize}
	\item Implementation af effektiv h�ndtering af xml-filen gemexport.xml.
	\begin{itemize}
		\item F�rste skridt er en analyse af strukturen, hvor resultatet er et xml skema, som kan validere vores xml fil. Dette g�res ved brug af BaseX,  hvor der oprettes en ny database p� grundlag af resource.xml.
		\item Flere forskellige foresp�rgelser skrives i hhv. Xpath og Xquery,  som bl.a. viser kardinalitetten af  elementer i xml'en. Samt om elementer er oprettet, men er tomme elementer. 
		\item Som udgangspunkt antages det at xml'en indeholder alle yderligheder der kan forekomme, s� f.eks. maks l�ngde for et tekst element kan udledes vha. Foresp�rgelser mod xml'en.
		\item Der inds�ttes en reference til det nyoprettede skema, i den oprindelige xml, og skemaet valideres med xml plugin i Notepad++. Dette kan selvf�lgelig g�res med et hvilken som helst xml valideringsv�rkt�j.

	\end{itemize}
	\item Nu skal xml'en gemmes i en relationel database, og her kan der v�lges flere tilgangsm�der.
	\begin{itemize}
		\item At modellere xml strukturen vha. Den Entitets og relationelle model, og splitte(shredde) xml op og inds�tte den, s� vi p� den m�de kan tilg� data og manipulere med den som en hvilken som helst anden sql database.
		\item vi kan v�lge at inds�tte hele vores xmlfil i en r�kke i en tabel, og dern�st prim�rt benytte Xpath foresp�rgelser til foresp�rgelser. Disse foresp�rgelser kan, og vil sikkert blive suppleret med brug af XmlSql funktioner for at strukturere resultatet af Xpath foresp�rgelserne yderligere.
		\item Sidste mulighed er delvis shredding, hvor f.eks.  de enkelte resourcer splittes ud i enkelte r�kker, med enkelte vigtige attributer, id, evt titel, trulket ud i egne kolonner, for at simplificere filtrering p� disse attributter.
	\end{itemize}
	\item Indl�sning foreg�r i postgresql 9.3, som kun underst�tter Xpath 1.0, dvs. Ingen Xquery og enkelte funktioner fra nutidig Xpath kan ikke benyttes. Derudover mangler postgresql underst�ttelse af xmlsql funktionen xmltable,  som ellers er oplagt til mapping fra xml til relationelle tabeller. Hvorimod de implementerede publishing functions er implementeret ok.
	\end{itemize}
\end{itemize}





