\chapter{Introduktion}
\label{chap:Introduktion}
\section{Form�l}
\label{sec:Formaal}
Denne rapport handler prim�rt om hvordan xml dokumenter kan valideres, uds�ttes for effektive foresp�rgelser og sameksistere med og i en SQL database. Alle emner bliver behandlet ved brug af et gennemg�ende eksempel, som l�bende uds�ttes for alle de forskellige teknologier som n�vnes herunder.
\begin{itemize}
	\item XML Skema
	\item XPath
	\item XQuery
	\item SQL/XML
	\item Full text search
\end{itemize}

\subsection{Fremgangsm�de}
\label{subsec:Fremgangsmaade}
Da det er generelt er umuligt at behandle et XML dokument, hvis ikke strukturen er kendt, skal der laves en mindre analyse af denne. Det er hvad kapitel \ref{chap:analyse_xml_struktur} omhandler, hvor \lang{XQuery} og \lang{XPath} benyttes. I kapitel \ref{chap:XML_skema_for_resource} udnyttes det at strukturen for xml dokumentet nu er defineret, og der oprettes et XML skema for at dokumentere dette p� en standard m�de, og dermed ogs� muligg�re validering af dokumentet.  Kapitel \ref{chap:XML_i_SQL_DB} handler om hvordan XML dokumenter kan indl�ses i en database, samt hvordan data kan udtr�kkes, og leveres som XML, uden brug af et applikationslag.  Tilsidst vil kapitel \ref{chap:fulltextserach} omhandle full text search, som muligg�re bedre s�gning i st�rre dokumenter og tekster.

Men f�r der kan startes p� alt dette, kommer f�rst en beskrivelse af det XML dokument, som rapporten benytter som datakilde.

\subsection{XML dokumentet resource}
\label{subsec:XML_dokumentet_resource}
XML dokumentet er gemt i filen \fil{resource.xml}, som er en kopi af filen \fil{gemexport.xml} fra kursets hjemmeside. Filen er at finde i mappen Data i projektmappen. Omd�bningen skyldes at navnet gemexport ikke var sigende for indholdet. 

Dataene i XML dokumentet er en liste over uddannelsesresourcer, hvor der for hver resource er en m�ngde oplysninger. Der er ud over titel og beskrivelse ogs� oplysninger om udgiver, kategorier og hvorn�r resourcen er oprettet og meget andet.  

I resten af rapporten vil brugen af navnet resource b�de referere til XML dokumentet, som er gemt i filen resource.xml, og til den instans af BaseX databasen der oprettes ud fra filen resource.xml. I tilf�lde hvor betydningen ikke fremg�r af sammenh�ngen, vil betydningen blive pr�ciseret eksplicit.




