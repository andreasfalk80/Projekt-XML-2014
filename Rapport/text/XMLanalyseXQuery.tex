 \chapter{Analyse af XML struktur}
\label{sec:analyse_xml_struktur}
At gennemskue, og udlede strukturen af et xml dokument kan v�re en stor opgave, is�r for et stort dokument med mange elementer. I tilf�ldet med \fil{resource.xml}, som indeholder utroligt meget data, kan strukturen ikke gennemskues ved en simpel genneml�sning af filen. 
Alternativet kan v�re at oprette en database i \prg{BaseX}, og ved brug af \lang{XQuery} forsp�rgelser kan der samles nok informationer om elementerne i xml'en, til at kunne beskrive strukturen. Denne tilgang kan forfines yderligere til en \lang{XQuery} funktion, der kan finde alle de �nskede oplysninger p� en gang. 
Derfor implementeres \lang{XQuery} funktionen \funk{local:analyze}, som netop returnerer metadata for xml strukturen. 

I figur \ref{code:struktur_metadata-xml} fremg�r det at der ud over elementernes navne ogs� findes evt. maks l�ngde og kardinalitet, samt at strukturen bibeholder for�ldre/barn forholdet mellem elementerne. For \xmlatt{maxlength} g�lder det at den kun udfyldes for elementer med tekstindhold, og betragter indholdet som en tekststreng, selvom det i praksis sagtens kunne v�re numerisk. Atributten \xmlatt{card} udfyldes med notationene \verb|1:1, 0:1, 0:n, m:n| og d�kker de normale muligheder for kardinalitet.
\begin{figure}[ht]
\centering
\begin{BVerbatim}
<elements>
  <element name="" maxlength="" card="">
    <element name="" maxlength="" card=""/>
	</element>
</elements>
\end{BVerbatim}
\caption{Struktur for genereret metadata-xml}
\label{code:struktur_metadata-xml}
\end{figure}

Funktionen \funk{local:analyze} returnere metadata om alle elementer, som er efterf�lgere til de elementer funktionen blev kaldt med, men ikke input elementerne selv. Det betyder at en analyse af et helt xml dokument, eller fragment, b�r startes med kaldet \verb|local:analyze(/)|, alts� med dokumentroden som parameter. 




Funktionen \funk{local:analyze} er en wrapper funktion til \funk{local:analyze2}, som er kodet som en rekursiv funktion der behanlder xml strukturen et niveau af gangen.




\begin{figure}[ht]
\centering
\begin{BVerbatim}
declare function local:analyze2($elements as item()*)
as element()*
{ 
   let $names :=  distinct-values($elements/*/name())
   for $el in $names 
   let  $maxlength := local:analyze_maxlength($elements/*[name()=$el])
   let $card := local:analyze_card($elements,$el)
   return 
   <element name='{$el}' maxlength='{$maxlength}' card='{$card}'>
       {local:analyze2($elements/*[name()=$el])}
   </element>
};
\end{BVerbatim}
\caption{Funktionerne 'local:analyze2'}
\label{code:func_analyze}
\end{figure}

\begin{figure}[ht]
\centering
\begin{BVerbatim}
<elements>
  <element name="root" maxlength="" card="1:1">
    <element name="resource" maxlength="" card="(1741)m:n(1741)">
      <element name="ID" maxlength="4" card="1:1"/>
      <element name="title" maxlength="107" card="1:1"/>
      <element name="description" maxlength="500" card="1:1"/>
      <element name="itemdate" maxlength="" card="1:1">
        <element name="recordcreated" maxlength="10" card="1:1"/>
        <element name="placedonline" maxlength="10" card="1:1"/>
      </element>
      <element name="identifier" maxlength="" card="1:1">
        <element name="url" maxlength="195" card="1:1"/>
      </element>
      <element name="publisher" maxlength="" card="1:1">
        <element name="name" maxlength="68" card="1:1"/>
        <element name="agency" maxlength="58" card="1:1"/>
      </element>
      <element name="image" maxlength="" card="1:1">
        <element name="url" maxlength="71" card="1:1"/>
        <element name="caption" maxlength="95" card="1:1"/>
        <element name="alttext" maxlength="200" card="1:1"/>
        <element name="sourceurl" maxlength="200" card="1:1"/>
      </element>
      <element name="interestingfact" maxlength="" card="1:1">
        <element name="url" maxlength="200" card="1:1"/>
        <element name="text" maxlength="523" card="1:1"/>
      </element>
      <element name="subjects" maxlength="" card="1:1">
        <element name="subject" maxlength="" card="(1)m:n(30)">
          <element name="category" maxlength="27" card="1:1"/>
          <element name="subcategory" maxlength="57" card="1:1"/>
          <element name="primary" maxlength="3" card="1:1"/>
        </element>
      </element>
      <element name="resourcekeywords" maxlength="" card="0:1">
        <element name="keywords" maxlength="" card="1:1">
          <element name="term" maxlength="40" card="(1)m:n(19)"/>
        </element>
      </element>
    </element>
  </element>
</elements>
\end{BVerbatim}
\caption{Resultat fra k�rsel af xml analyse}
\label{code:xml_analyse_result}
\end{figure}


