 \chapter{Analyse af resource}
\label{chap:analyse_xml_struktur}

\section{Form�l}
\label{sec:Formaal_xml_analyse}
At gennemskue, og udlede strukturen af et xml dokument kan v�re en stor opgave, is�r for et stort dokument med mange elementer. I tilf�ldet med resource, som indeholder utroligt meget data, kan strukturen ikke gennemskues ved en simpel genneml�sning af filen. 
Alternativet kan v�re at oprette en database i \prg{BaseX}, og ved brug af en r�kke \lang{XQuery} forsp�rgelser kan der samles nok informationer om elementerne i xml'en, til at kunne beskrive strukturen. P� denne m�de kan \prg{Basex} hj�lpe med at skabe overblik. Men det er stadig v�re en manuel process, hvor der l�bende udvikles nye foresp�rgelser efterh�nden som man opbygger kendskab til XML dokumentet. Figur \ref{xquery:count_empty_image} viser hvordan det kan afg�res om der findes elementer \xmlelm{image} hvor alle den under elementer er tomme. Hvis man �nsker at lave samme foresp�rgelse for elementet \xmlelm{itemDate}, s� kr�ver det at man skriver en helt ny foresp�rgelse.

\section{Design af analyse funktion}
\label{sec:DesignAfAnalyseFunktion}
Resultatet af analysen skal v�re en xml struktur, der beskriver de elementer som input xmlen best�r af. Figur \ref{code:struktur_metadata-xml} viser det �nskede resultat, hvor informationer om et element angives som attributter til elementet \xmlelm{element}, og informationer om dets attributer som \xmlelm{attribute} elementer efterfulgt af information om dets efterf�lgerer som angives som nestede elementer af samme type. **HUSK** eksempel som bilag hvor der rent faktisk er attributter. **HUSK**
\begin{figure}[ht]
\centering
\begin{BVerbatim}
<elements>
  <element name="" maxlength="" card="">
      <attribute name=""/>
    <element name="" maxlength="" card=""/>
	</element>
</elements>
\end{BVerbatim}
\caption{Struktur for genereret metadata-xml}
\label{code:struktur_metadata-xml}
\end{figure}

For \xmlatt{maxlength} g�lder det at den kun udfyldes for elementer med en text() node. Virkem�den for \xmlatt{maxlength} er ikke defineret for elementer med mixed-content, og da den p�g�ldende xml fra \fil{resource.xml} ikke indeholder mixed-content, bliver dette tilf�lde ikke behandlet yderligere.

Atributten \xmlatt{card} udfyldes med notationene \verb|1:1, 0:1, 0:n, m:n| og d�kker de normale muligheder for kardinalitet.

Funktionen \funk{local:analyze} begynder analysen ved input elementets b�rn, og returnerer derfor ingen oplysninger om input elementet. Det betyder at en analyse af et helt xml dokument, eller fragment, b�r startes med kaldet \verb|local:analyze(/)|, alts� med dokumentroden som parameter. Dette skyldes at f.eks. kardinalitets analysen kr�ver information om hvilken kontekst elementerne optr�der i.

Funktionen \funk{local:analyze} er en wrapper funktion til \funk{local:analyze2}, som er den rekursive funktion der reelt behandler xml strukturen. For hvert niveau skal alle tags med samme navn behandles og analyseres samlet, og derfor opbygges en liste af fundne elementnavne i linie \lref{04}.
Elementerne grupperes udfra deres navne for at g�re det muligt at udf�re en generel analyse, og dermed v�re i stand til at udlede oplysninger om element. finde alle mulige efterkommere, samt den maksimale l�ngde for en vilk�rlig text node under elementer med dette navn.
*TODO* beskriv at alle elementer med samme navn p� samme sted i strukturen betragtes som samme type, og derfor er en generel analyse vigtig/korrekt. *TODO*

Attributen \xmlatt{name} er simpel at finde v�rdien for, mens de to andre er en smule mere komplicerede. Derfor er der skrevet  hj�lpefunktionerne \funk{local:analyze\_maxlength} og \funk{local:analyze\_card}. Resultatet fra disse funktioner, som kaldes i linie \lref{06} og \lref{07}, tilf�jes xml strukturen som attributter. Den sidste funktion \funk{local:analyze\_attributes} finder alle de attributer der findes til elementet, samt deres maksimale l�ngde, og for hver attribut tilf�jes et element. *TODO* referer til indsat eksempel istedetfor at beskrive hvordan det tilf�jes til strukturen.... *TODO*  

\begin{figure}[ht]
\centering
\begin{BVerbatim}
01: declare function local:analyze2($elements as item()*)
02: as element()*
03: { 
04:    let $names :=  distinct-values($elements/*/name())
05:    for $el in $names 
06:    let  $maxlength := local:analyze_maxlength($elements/*[name()=$el])
07:    let $card := local:analyze_card($elements,$el)
08:    return 
09:    <element name='{$el}' maxlength='{$maxlength}' card='{$card}'>
10:        {local:analyze_attributes($elements/*[name()=$el])}
11:        {local:analyze2($elements/*[name()=$el])}
12:    </element>
13: };
\end{BVerbatim}
\caption{Den rekursive funktion 'local:analyze2'}
\label{code:func_analyze}
\end{figure}






















