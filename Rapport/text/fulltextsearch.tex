\chapter{Full Text Search}
\label{chap:fulltextserach}
\section{Form�l}
\label{sec:fulltextsearchformaal}
Det skal v�re muligt at lave fornuftig s�gning p� beskrivelser af resourcer.
Full text search er en teknik hvor det man s�ger i, og selve foresp�rgelsen bearbejdes inden et resultat kan pr�senteres. Denne bearbejdning er det der g�r at s�gningen i figur \ref{fts:simpel_eksempel} returnerer true, selvom ordet 'searches' ikke er i teksten der s�ges i.
En del af denne bearbejdning foreg�r ved at mappe en masse forskellige former af samme ord til et ord, f.eks. huse og hus, bliver betragtet som det samme. Der ud over er der en masse ord, som f.eks. og, da, n� som ikke giver mening at s�ge p� da de typisk optr�der i store m�ngder i teksterne. Disse stopord ignoreres.  

\begin{figure}[ht]
\centering
\begin{BVerbatim}
select 
  to_tsvector('english','This is a test of full text search') 
  @@ 
  to_tsquery('english', 'searches');
------------------------------------
	                     t
\end{BVerbatim}
\caption{Simpelt eksempel p� full text search}
\label{fts:simpel_eksempel}
\end{figure}

Der implementeres full text search p� feltet resource.description i den modellerede SQL udgave af resource dokumentet. Det g�res ved at implementere en funktion der tager et ord som input, og s� returnere udvalgte kolonner fra de resourcer, hvor ordet er fundet. 


\section{Implementation af full text search}
\label{sec:impl_af_FTS}

Ved udf�relsen af en s�gning, skal begge funktioner to\_tsvector og to\_tsquery udf�res, og dette kan, med fordel, optimeres i tilf�lde hvor der skal s�ges ofte, eller i meget store dokumenter.

For at g�re det muligt at indeksere teksten, kan et indeks oprettes, hvor det er funktionen to\_tsvector der indekseres. En anden fremgangsm�de er at tilf�je en kolonne til tabellen, som kan indeholde to\_tsvector resultatet af det eller de felter der skal kunne s�ges i. I projektet er den sidste fremgangsm�de valgt. Der er fordele og ulemper ved begge fremgangsm�der, og ved den der er valgt her, er der ekstra pladsforbrug til at gemme ts\_vectoren og at den skal holdes opdateres hver gang description feltet opdateres. Og det er netop p� grund af behovet for denne synkronisering at metoden er valgt, for det er en oplagt mulighed for at implementere en trigger. Og selvom Postgresql har en funktion indbygget 'tsvector\_update\_trigger' som man kan angive som trigger, implementeres funktionen fra grunden.

S� f�lgende opgaver skal l�ses:

\begin{compactitem}
\item En triggerfunktion skal kodes. Den skal opdatere v�rdien af ts\_vectoren, s� den svarer til description feltet.
\item Tabellen resource skal udvides med den nye kolonne.
\item En trigger skal oprettes, som anvender den ny udviklede triggerfunktion.
\end{compactitem}
\vspace{1.0mm}
En simpel funktion implementeres, hvor funktionen to\_tsvector benyttes p� description, og tildeles til feltet FTS\_index\_col.
\begin{figure}[ht]
\centering
\begin{BVerbatim}
drop function if exists resource_trigger();
create function resource_trigger() returns trigger
as $$
declare
begin
  new.FTS_index_col := 
  to_tsvector('english', new.description);
return new;
end
$$
language plpgsql;
\end{BVerbatim}
\caption{Triggerfunktion}
\label{fts:trigger_func}
\end{figure}

Triggerfunktionen skal nu kobles til en trigger, og det g�res med f�lgende stykke kode. 
\begin{figure}[ht]
\centering
\begin{BVerbatim}
create trigger resource_FTS_index_col 
  before update of description,title or insert 
  on resource 
  for each row
  execute procedure resource\_trigger();
\end{BVerbatim}
\caption{Oprettelse af trigger}
\label{fts:trigger_create}
\end{figure}

Med alt dette defineret, plus et indeks oprettet p� den nye kolonne, kan der foresp�rges p� indholdet i description feltet. Funktionen searchDescription returnerer id, title og selve beskrivelsen for alle de r�kker der matcher det angivede ord.
\begin{figure}[ht]
\centering
\begin{BVerbatim}
drop function if exists searchDescription(text);
create function searchDescription(word text) returns 
  TABLE(resourceid int,title text, description text)
as $$
declare
begin
return query 
select r.resourceid,r.title,r.description
FROM resource as r
WHERE fts_index_col @@ to_tsquery('english', word);
end
$$
language plpgsql;
\end{BVerbatim}
\caption{Funktionen searchDescription}
\label{fts:searchDescription}
\end{figure}









