\begin{itemize}
\item[simpleType og restriction] Typen titleType tager udgangspunkt i typen string, men tilf�jer en begr�nsning p� l�ngen af strengen, ved brug af minLength og maxLength. UrlType bruger facetten pattern, til at angive et simpelt format tjek af en url adresse, og den sidste simple type, primaryType bruger facetten enumeration, til at begr�nse v�rdierne til YES og NO.
%Der m� ikke rettes i formattering, s� g�r det galt. Kode b�r kopieres over noteblok, for at fjerne ekstra formateringen fra Notepad++
\begin{figure}[ht]
\centering
\begin{BVerbatim} 
<xs:simpleType name='urlType'>
  <xs:restriction base='xs:string'>
    <xs:pattern value="((http|https)?://.*)?" />
  </xs:restriction>
</xs:simpleType>
\end{BVerbatim}
\caption{Eksempel p� en simpel type, med en begr�nsning som anvender et pattern}
\label{code:def_urlType}
\end{figure}

\item[standard typer] Enkelte elementer benytter sig af standard typer, f.eks. ID som defineres som positiveInteger, og description som en string uden begr�nsninger, samt recordcreated og placedonline i itemDate, der begge bruger typen date. 
\end{itemize}



%Der m� ikke rettes i formattering, s� g�r det galt. Kode b�r kopieres over noteblok, for at fjerne ekstra formateringen fra Notepad++
\begin{figure}[ht]
\centering
\begin{BVerbatim} 
<xs:complexType name='resourcekeywordsType'>
  <xs:sequence>
    <xs:element name='keywords'>
      <xs:complexType>
        <xs:sequence>
	  <xs:element name='term' 
                      type='xs:string' 
                      maxOccurs='unbounded'/>
        </xs:sequence>
      </xs:complexType>
    </xs:element>
  </xs:sequence>
</xs:complexType>
\end{BVerbatim}
\caption{Eksempel p� navngivet og anonym type, der samtidig er kompleks}
\label{code:def_resourcekeywordsType}
\end{figure}
