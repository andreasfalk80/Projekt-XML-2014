\chapter{XML i SQL database}
\label{chap:XML_i_SQL_DB}
\section{Form�l}
Det er muligt at gemme xml dokumenter i en SQL database, og til form�let er der lavet en ny SQL datatype der hedder xml. Ved at gemme XML i en kolonne af denne type, kan der laves foresp�rgelser via \lang{XPath} og \lang{XQuery}, i sammenspil med en normal SQL foresp�rgelse. Denne funktionalitet kaldes ogs� SQL/XML.

XML dokumentet resource skal nu gemmes i en relationel database, n�rmere bestemt \prg{PostgreSQL~9.3}.%, som pt. kun underst�tter \lang{XPath 1.0} foresp�rgelser. 
F�r et XML dokument kan gemmes i en sql database, b�r man overveje hvilken strategi der skal f�lges. 
\begin{itemize}
	\item XML dokumentet gemmes i en enkelt celle i en tabel, uden der foretages nogen som helst bearbejdning, og SQL databasen benyttes udelukkende som et opbevaringssted.
	\item XML dokumentet splittes(shreddes), og inds�ttes i en datamodel der er modelleret til at passe med den oprindelige XML struktur. 
	
	Man kan foretage enten en fuldst�ndig eller delvis opsplitning. 
	
	I tilf�ldes med resource, kunne en delvis opsplitning v�re at hvert \xmlelm{resource} element inds�ttes i en r�kke best�ende  af v�rdien fra \xmlelm{ID} trukket ud, og lagt i en kolonne, samt en kolonne af typen xml, der indeholder xml fragmenten for den enkelte resource.
	
	En fuld opsplitning vil betyder at der ikke gemmes XML, da al data er remodeleret.
\end{itemize}

I de n�ste afsnit vil XML dokumentet resource blive gemt i en relationel database, det ske ved hj�lp af forskellige \lang{XPath} foresp�rgelser. Derefter vil forskellige SQL/XML funktioner blive brugt til at opbygge XML dokumenter udfra den relationelle database.

\section{Design af XML dokumentet resource i relationel database}
\label{sec:Design_af_XML_resource_rel_db}
I forbindelse med denne rapport v�lges den totale opsplitning af XML dokumentet til den relationelle datamodel, af den simple grund at det giver st�rst mulighed for at pr�ve kr�fter med SQL/XML og \lang{XPath}.

Ud fra skemaet for resource, kan man udfra kardinaliteten for elementerne f� et udgangspunkt for hvilke entiteter der er behov for. I figur \ref{fig:ER_diagram} kan den model som er fremkommet ses. Der vil ikke blive forklaret hvilke beslutninger der ligger til grund for modellen, da den er af sekund�r betydning i denne rapport. 


\begin{figure}[ht]
  \centering
   \framebox{\includegraphics[width=1\textwidth]{pic/gemexport_ER.pdf}}
   \caption{ER diagram for resource database}
   \label{fig:ER_diagram}
\end{figure}


